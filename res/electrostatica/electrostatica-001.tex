% electrostatica-001.tex
%
% Copyright (C) 2019-2025 José A. Navarro Ramón <janr.devel@gmail.com>
% .............................................................................
% OBSERVACIÓN: Se puede dar formato al búffer en AUCTeX con: M-x L-buff RET
% .............................................................................

\documentclass[a4paper,10pt]{article}

\usepackage{../edclasica-res.pkg}
\usepackage{../edclasica-res.defs}

% *****************************************************************************
% ******* DEFINICIONES DE ESTE EJERCICIO **************************************
% *****************************************************************************
% Bloques de ejercicios
\renewcommand*{\mainsubject}{Electrostática}
\renewcommand*{\parte}{EJERCICIOS DE ELECTRODINÁMICA CLÁSICA}
\renewcommand*{\tipoBloque}{Problema}
\renewcommand*{\bloque}{2}
\renewcommand*{\hoja}{1}
\renewcommand*{\ejBloque}{1}
% Fuente: examen
%\renewcommand*{\ejExamen}{1}
\renewcommand*{\fuente}{Introduction to Electrodynamics - Griffiths - 4th Ed.
  Exercise 2.1. Pages 60-61.}

% *****************************************************************************

\begin{document}

% ############################ ENUNCIADO ######################################
% Entrada en el índice del fichero 'pdf' 'enunciado.0'
\pdfbookmark[0]{Enunciado}{enunciado}

% -----------------------------------------------------------------------------
% 
% -----------------------------------------------------------------------------
\begin{qboxshort}
\begin{subejercicio}
  \item  Se sitúan doce cargas iguales $q$ en los vértices de un polígono
    regular de doce lados (dodecágono).
    ¿Cuál es la fuerza neta sobre una carga de prueba $Q$ en el centro
    del polígono?
  \item Suponga que se elimina una de las doce cargas $q$.
    ¿Cuánto vale ahora la fuerza neta sobre $Q$?
  \item Ahora tenemos trece cargas iguales $q$ sobre los vértices de un
    polígono regular de trece aristas. ¿Cuál es la fuerza neta sobre una
    carga de prueba $Q$ en el centro del polígono?
  \item Si se retira una de las trece cargas $q$, ¿cuál sería ahora la
    fuerza neta sobre $Q$?
\end{subejercicio}
\end{qboxshort}

% ######################### RESOLUCIÓN ########################################

% ================ APARTADO a) ==============================================
% Entrada en el índice del fichero 'pdf'
\pdfbookmark[0]{Apartado a}{a}

\begin{soluc}
\item A la izquierda de la figura~\ref{fig:cargas_en_dodecagono} se representan
  doce cargas iguales numeradas\footnotemark{}
  \footnotetext{Se podrían numerar del 1 al 12. En este caso se hace así para
    que la numeración coincida con la de la segunda demostración.}
  $q_{0} = q_{1} = \cdots = q_{11} = q$, situadas sobre los vértices de un
  polígono regular de doce lados y una carga de prueba $Q$ en su centro.
  La conclusión a la que llegaremos no dependerá del signo de las cargas, de
  manera que supondremos que $q$ y $Q$ tienen el mismo signo, así la fuerza
  eléctrica de entre las cargas es repulsiva (en caso contrario sería
  atractiva).
  A la derecha de la misma figura se representan las fuerzas (en este caso,
  repulsivas) entre cada carga $q_{k}$ y la carga de prueba central $Q$.
  Todas ellas tienen el mismo módulo
  \[
    F_{k}
    = |\vvv{F}_{k}|
    = \dfrac{1}{4\pi\epsilon_{0}} \dfrac{Q q_{k}}{r_{k}^{2}}
    = \dfrac{1}{4\pi\epsilon_{0}} \dfrac{Q q}{r^{2}}
    = \text{constante}
    \hspace{2em}k=0,1,2,\cdots 11
  \]
  \begin{figure}[ht]
    \def\scl{1}
    \def\lado{2.5}
    \def\qsize{7.5pt}
    \def\Qsize{6.5pt}
    \centering
    \begin{minipage}{0.45\linewidth}
      \begin{tikzpicture}[scale=\scl]
        \coordinate (m0) at (0:\lado);
        \foreach[remember=\p as \pold (initially 0)]
        \p in {1,...,11,0}{
          \coordinate (m\p) at (30*\p:\lado);
          \draw[ultra thin,black!20] (m\pold) -- (m\p);
          \filldraw[fill=yellow!30,draw=black!40] (m\pold) circle[radius=\qsize];
          \node at (m\pold) {\footnotesize $q_{\pold}$};
        }
        \filldraw[fill=yellow!30,draw=black!40] (m0) circle[radius=\qsize];
        \node at (m0) {\footnotesize $q_{0}$};
        \filldraw[fill=green!80,draw=green!40!black] (0,0) circle [radius=\Qsize];
        \node at (0,0) {\footnotesize$Q$};
      \end{tikzpicture}
    \end{minipage}
    \hspace{2em}
    \begin{minipage}{0.45\linewidth}
      \begin{tikzpicture}[scale=\scl]
        \coordinate (m0) at (0:\lado);
        \foreach[evaluate=\p as \f using {int(mod(\p+6,12))},
        remember=\p as \pold (initially 0)] \p in {1,2,...,11,0}{
          \coordinate (m\p) at (30*\p:\lado);
          \draw[ultra thin,black!20] (m\pold) -- (m\p);
          \filldraw[fill=yellow!30,draw=black!40] (m\pold) circle[radius=\qsize];
          \node at (m\pold) {\footnotesize $q_{\pold}$};
          \draw[-{Latex[round]},shorten >=10mm] (0,0) -- (m\p)
          node[pos=0.70] {\footnotesize $\vvv{F}_{\f}$};
        }
        \filldraw[fill=yellow!30,draw=black!40] (m0) circle[radius=\qsize];
        \node at (m0) {\footnotesize $q_{0}$};
        \filldraw[fill=green!80,draw=green!40!black] (0,0) circle [radius=\Qsize];
        \node at (0,0) {\footnotesize$Q$};
      \end{tikzpicture}
    \end{minipage}
    \caption{Doce cargas iguales $q_{0}= q_{1}= \cdots = q_{11} = q$ en amarillo,
      situadas sobre los vértices de un dodecágono regular y una carga de prueba
      $Q$ en verde, en su centro. Si se supone que las cargas $q$ y $Q$ tienen el
      mismo signo, las fuerzas serían repulsivas y tienen el mismo módulo.
      Si en cambio, fueran de signo contrario, serían fuerzas atractivas
      (habría que redibujarlas, pero seguirían teniendo el mismo módulo.}
    \label{fig:cargas_en_dodecagono}
  \end{figure}

Cuando el polígono regular tiene un número par de lados, como en este caso,
se aprecia claramente que cada una de las fuerzas que actúan sobre la carga de
prueba se corresponde con otra fuerza opuesta --véanse, por ejemplo
$\vvv{F}_{6}$ y $\vvv{F}_{0}$--, dando en total una resultante $\vvv{R}$ nula
\begin{equation}\label{eq:equilibrioA}
  \vvv{R} = \sum_{k=0}^{11} \vvv{F}_{k} = 0
\end{equation}

 %\bigskip
  % ================ APARTADO b) ==============================
  % Entrada en el índice del fichero 'pdf'
  \pdfbookmark[0]{Apartado b}{b}
\item Si se elimina una de las doce cargas, por ejemplo $q_{11}$, la resultante
  de las fuerzas repulsivas que las restantes cargas ejercen sobre $q$ sería
  $\vvv{R}' = \sum_{k=0}^{10} \vvv{F}_{k}$
  Partimos de la expresión de equilibrio \eqref{eq:equilibrioA} del apartado
  anterior y separamos la fuerza $\vvv{F_{11}}$
  \[
    \vvv{R}
    = \sum_{k=0}^{11} \vvv{F}_{k}
    = \sum_{k=0}^{10} \vvv{F}_{k} + \vvv{F}_{11}
    = \vvv{R}' + \vvv{F}_{11}
    = 0
  \]
  Despejando la suma de las fuerzas que actúan sobre la carga de prueba
  demostramos que la nueva resultante es la opuesta de la fuerza que ejercería
  la carga que hemos eliminado, que en nuestro ejemplo es
  $-\vvv{F}_{11} = \vvv{F}_{5}$
  \[
    \vvv{R}' = \sum_{k=0}^{10} \vvv{F}_{k} = -\vvv{F}_{11} = \vvv{F}_{5}
  \]
  
%\bigskip
% ================ APARTADO c) ==============================
% Entrada en el índice del fichero 'pdf'
  \pdfbookmark[0]{Apartado c}{c}
\item
  Ahora las cargas idénticas $q$ se encuentran distribuidas sobre los vértices
  de un polígono regular de trece lados, como se observa en la
  figura~\ref{fig:cargas_en_tridecagono}. Supondremos como en el apartado
  anterior que las cargas tienen el mismo signo --aunque la conclusión
  final no dependerá del signo de las cargas--. Las fuerzas repulsivas se
  representan en la misma figura.
  
  \begin{figure}[ht]
    \def\scl{1}
    \def\lado{2.5}
    \def\qsize{7.5pt}
    \def\Qsize{6.5pt}
    \centering
    \begin{minipage}{0.45\linewidth}
    \begin{tikzpicture}[scale=\scl]
      \coordinate (m0) at (0:\lado);
      \foreach[remember=\p as \pold (initially 0)]
      \p in {1,...,12,0}{
        \coordinate (m\p) at (360*\p/13:\lado);
        \draw[ultra thin,black!20] (m\pold) -- (m\p);
        \filldraw[fill=yellow!30,draw=black!40] (m\pold) circle[radius=\qsize];
        \node at (m\pold) {\footnotesize $q_{\pold}$};
    }
    \filldraw[fill=yellow!30,draw=black!40] (m0) circle[radius=\qsize];
    \node at (m0) {\footnotesize $q_{0}$};
    \filldraw[fill=green!80,draw=green!40!black] (0,0) circle [radius=\Qsize];
    \node at (0,0) {\footnotesize$Q$};
  \end{tikzpicture}
\end{minipage}
\hspace{2em}
\begin{minipage}{0.45\linewidth}
  \begin{tikzpicture}[scale=\scl]
      \coordinate (m0) at (0:\lado);
      \foreach[evaluate=\p as \f using {int(mod(\p+6,13))},
      remember=\p as \pold (initially 0)] \p in {1,2,...,12,0}{
        \coordinate (m\p) at (360*\p/13:\lado);
        \draw[ultra thin,black!20] (m\pold) -- (m\p);
        \filldraw[fill=yellow!30,draw=black!40] (m\pold) circle[radius=\qsize];
        \node at (m\pold) {\footnotesize $q_{\pold}$};
        \draw[-{Latex[round]},shorten >=10mm] (0,0) -- (m\p)
             node[pos=0.70] {\footnotesize $\vvv{F}_{\f}$};
    }
    \filldraw[fill=yellow!30,draw=black!40] (m0) circle[radius=\qsize];
    \node at (m0) {\footnotesize $q_{0}$};
    
    \filldraw[fill=green!80,draw=green!40!black] (0,0) circle [radius=\Qsize];
    \node at (0,0) {\footnotesize$Q$};
    \end{tikzpicture}
  \end{minipage}

\caption{Trece cargas iguales $q_{0}= q_{1}= \cdots = q_{12} = q$ en amarillo,
  situadas sobre los vértices de un polígono regular de trece lados, junto con
  una carga de prueba $Q$ en verde, en su centro.
  Si se supone que las cargas $q$ y $Q$ tienen el mismo signo, las fuerzas
  serían repulsivas y tienen el mismo módulo. Si fueran de distinto signo, las
  fuerzas serían atractivas pero tambíen tendrían el mismo módulo.}
\label{fig:cargas_en_tridecagono}
\end{figure}

A diferencia del caso con un número par de cargas, cada fuerza no tiene
una opuesta que la anule, por lo que no podemos utilizar el razonamiento anterior.
Utilizaremos dos demostraciones diferentes:

\begin{enumerate}
\item En primer lugar utilizaremos  un razonamiento --también aplicable al primer
apartado del ejercicio-- basado en la simetría.
Podemos afirmar que
\begin{quotation}
  \emph{Como las fuerzas ejercidas sobre la carga de prueba $Q$ están
    distribuidas simétricamente en torno al centro del polígono,
    la resultante debe ser nula.}
\end{quotation}

Para convencernos de lo anterior recordemos primero que hemos etiquetado
las cargas $q$ en los vértices y las fuerzas $\vvv{F}$ sobre la carga de
prueba para distinguirlas, pero que valen lo mismo en el caso de las cargas
y las fuerzas tienen el mismo módulo.

Ahora supondremos que la resultante no se anula. Entonces, por ser un vector
no nulo, $\vvv{F}\ne 0$, debería tener una cierta dirección y sentido.
Demostraremos que esto nos conduce a una contradicción por lo que
concluiremos que la resultante se debe anular.

En la izquierda de la figura~\ref{fig:cargas_en_tridecagono_sin_etiquetar}
representamos las cargas y las fuerzas en la configuración original pero
sin etiquetarlas.

  \begin{figure}[ht]
    \def\scl{1}
    \def\lado{2.5}
    \def\qsize{7.5pt}
    \def\Qsize{6.5pt}
    \centering
\begin{minipage}{0.45\linewidth}
  \begin{tikzpicture}[scale=\scl]
      \coordinate (m0) at (0:\lado);
      \foreach[evaluate=\p as \f using {int(mod(\p+6,13))},
      remember=\p as \pold (initially 0)] \p in {1,2,...,12,0}{
        \coordinate (m\p) at (360*\p/13:\lado);
        \draw[ultra thin,black!20] (m\pold) -- (m\p);
        \filldraw[fill=yellow!30,draw=black!40] (m\pold) circle[radius=\qsize];
        \node at (m\pold) {\footnotesize $q$};
        \draw[-{Latex[round]},shorten >=10mm] (0,0) -- (m\p)
             node[pos=0.70] {\footnotesize $\vvv{F}$};
    }
    \filldraw[fill=yellow!30,draw=black!40] (m0) circle[radius=\qsize];
    \node at (m0) {\footnotesize $q$};
    
    \filldraw[fill=green!80,draw=green!40!black] (0,0) circle [radius=\Qsize];
    \node at (0,0) {\footnotesize$Q$};
    \end{tikzpicture}
  \end{minipage}
\hspace{2em}
    \begin{minipage}{0.45\linewidth}
    \begin{tikzpicture}[scale=\scl]
      \coordinate (m0) at (0:\lado);
      \foreach[remember=\p as \pold (initially 0)]
      \p in {1,...,12,0}{
        \coordinate (m\p) at (360*\p/13:\lado);
        \draw[ultra thin,black!20] (m\pold) -- (m\p);
        \filldraw[fill=yellow!30,draw=black!40] (m\pold) circle[radius=\qsize];
        \node at (m\pold) {\footnotesize $q$};
      }
    \filldraw[fill=yellow!30,draw=black!40] (m0) circle[radius=\qsize];
    \node at (m0) {\footnotesize $q$};

    \draw[-{Latex[round]},shorten >=10mm] (0,0) -- (65:\lado)
         node[pos=0.70] {\footnotesize $\vvv{R}$};
    
    \filldraw[fill=green!80,draw=green!40!black] (0,0) circle [radius=\Qsize];
    \node at (0,0) {\footnotesize$Q$};
  \end{tikzpicture}
\end{minipage}

\caption{Trece cargas iguales $q$ en amarillo, situadas sobre los vértices de
  un polígono regular de trece lados, junto con una carga de prueba $Q$ en
  verde, en su centro. La fuerzas sobre la carga  de prueba tendrían el mismo
  módulo --ahora hemos supuesto que las cargas $q$ y $Q$ tienen distinto signo,
  por lo que las fuerzas son atractivas, aunque esto no afecta a nuestra
  conclusión--. A la derecha representamos las cargas y una hipotética fuerza
  resultante no nula sobre $Q$.}
\label{fig:cargas_en_tridecagono_sin_etiquetar}
\end{figure}

Nos damos cuenta de que la figura tiene un eje de simetría de rotación de
orden 13, esto es, si rotamos la figura un ángulo múltiplo de $\ang{360}/13$ la
configuración resultante sería completamente indistiguible de la original.
La resultante de estas fuerzas debe cumplir esta simetría.

A la derecha de la figura representamos las cargas y la resultante de las fuerzas
que actúan sobre la carga de prueba, que hemos supuesto que no es nula y tiene
por tanto una cierta dirección y sentido que elegimos al azar. 
Pero si giramos la figura de la derecha según el eje de simetría de la figura
deberíamos llegar a una situación indistinguible de la original, pero esto
no ocurre con la resultante $\vvv{R}$ que hemos dibujado, pues nos daríamos cuenta
de que se ha girado. De hecho, no importa cómo representemos la resultante no
nula; nunca cumplirá con la simetría del sistema.
La única manera de que la resultante obedezca la simetría del sistema es que
se anule
\[
  \vvv{R} = \sum_{k=0}^{13} F_{k} = 0
\]

\item Ahora demostraremos que la resultante es nula de una forma más elaborada
  y prescindible, habida cuenta de la sencillez del argumento basado en la
  simetría expuesto en el punto anterior. Para cambiar algo, seguiremos
  a un caso general de $N$ cargas situadas en los vértices de un polígono
  regular de $N$ lados, aunque cuando realicemos un dibujo representaremos
  un pentángono regular como modelo de polígono regular general de $N$ lados
  
\begin{figure}[ht]
    \def\scl{1}
    \def\lado{2}
    \def\qsize{7.5pt}
    \def\Qsize{6.5pt}
    \centering
\begin{minipage}{0.5\linewidth}
  \begin{tikzpicture}[scale=\scl]
    \node at (-3.6,0) {$\phantom{1}$};
      \coordinate (O) at (0,0);
      \coordinate (m0) at (0:\lado);
      \foreach[evaluate=\p as \f using {int(mod(\p+6,5))},
      remember=\p as \pold (initially 0)] \p in {1,2,...,4,0}{
        \coordinate (m\p) at (360*\p/5:\lado);
        \draw[ultra thin,black!20] (m\pold) -- (m\p);
        \draw[-{Latex[round]},shorten >=0pt] (0,0) -- (m\p);
    }

    \pic [draw, shorten >=1pt, shorten <=1pt, ->, "$\frac{2\pi}{N}$",
    angle eccentricity=1.8]
    {angle = m0--O--m1};
    
    \path (0,0) -- (m0) node[pos=1.3] {$k=0$};
    \path (0,0) -- (m1) node[pos=1.3] {$k=1$};
    \path (0,0) -- (m2) node[pos=1.3] {$k=2$};
    \path (0,0) -- (m3) node[pos=1.3] {$\cdots$};
    \path (0,0) -- (m4) node[pos=1.3] {$k=N-1$};
    
    \filldraw[fill=green!80,draw=green!40!black] (O) circle [radius=\Qsize];
    \node at (O) {\footnotesize$Q$};
  \end{tikzpicture}
    \caption{En esta figura se representan las $N$ fuerzas $\vvv{F}_{k}$ sobre la
    carga de prueba utilizando un pentágno regular como modelo de polígono
    regular de $N$ lados. El ángulo entre dos fuerzas adyacentes es $2\pi/N$.
  En el diagrama las fuerzas se numeran con $k=0,1,\cdots,N-1$.}
  \label{fig:cargas_en_poligono_regular}
  \end{minipage}
\end{figure}

En el diagrama tenemos $N$ fuerzas $\vvv{F}_{k}$, con $k=0,1,\cdots,N-1$.
El ángulo de la fuerza $\vvv{F}_{k}$ con el eje de abcisas (horizonal) es
\[
  \theta_{k} = k\,\frac{2\pi}{N}
  \hspace{2em}\text{con }
  k = 0,1,\cdots, N-1
\]

Cada fuerza $\vvv{F}_{k}$ se debe a la interacción de la carga $q_{k}$ con $Q$.
Esta interacción puede ser atractiva o repulsiva. Se debe hacer hincapié en
que el polígono regular representado es el que forman los extremos de los
vectores fuerza $\vvv{F}_{k}$ y no necesariamente el que forman las cargas
$q_{k}$.

La fuerza que ejerce $q_{k}$ sonbre la carga de prueba $Q$ es
\[
  \vvv{F}_{k}
  = \frac{1}{4\pi\epsilon_{0}} \,\frac{Q\,q_{k}}{r^{2}}\,\xhat{u}_{k}
  = \frac{1}{4\pi\epsilon_{0}}\,\frac{Q\,q}{r^{2}}
    \left[
      \cos\left(\frac{2\pi}{N}\right)\,\xhat{u}_{x}
      + \sin\left(\frac{2\pi}{N}\right)\,\xhat{u}_{y}  
    \right]
\]
donde $r$ es la distancia entre cada $q_{k}$ y la carga de prueba y
$\xhat{u}_{k}$ es el vector unidad que tiene la misma dirección y
sentido que $\vvv{F}_{k}$

\begin{figure}[ht]
  \def\scl{1}
  \def\longejex{3}
  \def\longejey{2.3}
  \def\longvector{2.5}
  \centering
  \begin{tikzpicture}[scale=\scl]
    \coordinate (O) at (0,0);
    \coordinate (x) at (\longejex,0);
    \coordinate (y) at (0,\longejey);
    \coordinate (uk) at (35:\longvector);
    \coordinate (cortex) at (uk |- x);
    \coordinate (cortey) at (uk -| y);

    \pic [draw, shorten >=1pt, shorten <=1pt, ->, "$\theta$",
    angle eccentricity=1.6]
    {angle = x--O--uk};
    
    \draw[-{Latex}] (O) -- (x) node[right] {\small $x$};
    \draw[-{Latex}] (O) -- (y)  node[left] {\small $y$};
    \draw[-{Latex[round]}, ultra thick] (O) -- (uk) node[above]
    {\small $\xhat{u}_{x}$};

    \draw[black!20, ultra thin] (uk) -- (cortex);
    \draw[black!20, ultra thin] (uk) -- (cortey);

    \path (O) -- (cortex) node[midway,below] {$\cos\theta$};
    \path (O) -- (cortey) node[midway,left] {$\sin\theta$};

    \node at (7,1)
    {$\xhat{u}_{k} = \cos\theta_{k}\,\xhat{u}_{k}
      + \sin\theta_{k}\,\xhat{u}_{kxs}$};
  \end{tikzpicture}
\end{figure}

La fuerza neta sobre la carga de prueba $Q$ es
\[
  \vvv{R}
  =
  \sum_{k=0}^{N-1} \vvv{F}_{k}
  =
  \frac{Q\,q}{4\pi\epsilon_{0}} \sum_{k=0}^{N-1}
  \left[
    \cos\left(\frac{2\pi k}{N}\right)\,\xhat{u}_{x}
    +
    \sin\left(\frac{2\pi k}{N}\right)\,\xhat{u}_{y}
    \right]
\]

\begin{equation}\label{eq:resultante}
  \vvv{R}
  =
    \frac{Q\,q}{4\pi\epsilon_{0}}
    \left\{
    \left[\sum_{k=0}^{N-1}\cos\left(\frac{2\pi k}{N}\right)\right]\,\xhat{u}_{x}
    +
    \left[\sum_{k=0}^{N-1}\sin\left(\frac{2\pi k}{N}\right)\right]\,\xhat{u}_{y}
    \right\}  
\end{equation}

Tenemos que calcular los dos sumatorios
\[
  \sum_{k=0}^{N-1} \cos\left(\frac{2\pi k}{N}\right)
  \hspace{1em}
  \text{y}
  \hspace{1em}
  \sum_{k=0}^{N-1} \sin\left(\frac{2\pi k}{N}\right)
\]

Utilizaremos el cálculo complejo. Las $\xhat{u}_{k}$ se pueden representar
por las $N$ diferentes raíces $N$-ésimas de la unidad, $\omega_{k}$ en el plano
complejo
\[
  1 = e^{i2\pi}
  \,\text{;}
  \hspace{1em}
  \omega_{k}
  =
  \sqrt[N]{1}
  =
  e^{i\,2\pi k/N}
  \,\text{;}
  \hspace{1em}
  k = 0,1,\cdots,N-1
\]

Nótese que
\[
  \omega_{k}^{N} = e^{i\frac{2\pi k}{\cancelout{N}}\,\cancelout{N}} = 1
\]

La raíz $\omega_{k}$ se puede escribir como una potencia, llamando
$\omega = \omega_{1} = e^{i2\pi/N}$
\[
  \omega_{k}
  =
  e^{i2\pi k/N}
  =
  \left(e^{i2\pi/N}\right)^{k}
  =
  \omega^{k}
\]

Sumamos las $N$ raíces
\[
  S
  =
  \sum_{k=0}^{N-1} \omega_{k}
  =
  \sum_{k=0}^{N-1} \omega^{k}
\]

Multiplicamos la expresión anterior por $\omega$
\[
  S\,\omega = \sum_{k=0}^{N-1} \omega^{k}\,\omega
\]

Restamos $S$
\[
  S\,\omega - S = \sum_{k=0}^{N-1} \omega^{k}\,\omega - \sum_{k=0}^{N-1} \omega^{k}
  =
  \left(\cancelout{\omega}+\cancelout{\omega^{2}}+\cdots+\cancelout{\omega^{N-1}}
    +\omega^{N}\right)
  - 1-\cancelout{\omega}-\cancelout{\omega^{2}}-\cdots-\cancelout{\omega^{N-1}}
  =
  \omega^{N} - 1
\]

Despejamos $S$
\[
  S
  = \frac{\omega^{N} - 1}{\omega - 1}
  = \frac{1 - 1}{\omega - 1}
  = 0
\]

Por tanto
\[
  S
  =
  \sum_{k=0}^{N-1} e^{i2\pi k/N}
  =
  \sum_{k=0}^{N-1}
  \left[\cos\left(\frac{2\pi k}{N}\right)
    + i \sin\left(\frac{2\pi k}{N}\right)\right]
  =
  \left[\sum_{k=0}^{N-1} \cos\left(\frac{2\pi k}{N}\right)
    + i \sum_{k=0}^{N-1} \sin\left(\frac{2\pi k}{N}\right)\right]
  =
  0 + 0\,i
\]

Así
\[
  \sum_{k=0}^{N-1} \cos\left(\frac{2\pi k}{N}\right)
  =
  \sum_{k=0}^{N-1} \sin\left(\frac{2\pi k}{N}\right)
  = 0
\]

Finalmente, deducimos de la ecuación~(\ref{eq:resultante}), que la
resultante se anula
\[
  \vvv{R}
  =
    \frac{Q\,q}{4\pi\epsilon_{0}}
    \left\{
    \left[\sum_{k=0}^{N-1}\cos\left(\frac{2\pi k}{N}\right)\right]\,\xhat{u}_{x}
    +
    \left[\sum_{k=0}^{N-1}\sin\left(\frac{2\pi k}{N}\right)\right]\,\xhat{u}_{y}
    \right\}
  =
  \frac{Q\,q}{4\pi\epsilon_{0}}\,0 = 0
\]

\end{enumerate}


\bigskip
% ================ APARTADO d) ==============================
% Entrada en el índice del fichero 'pdf'
\pdfbookmark[0]{Apartado d}{d}
\item De forma similar a como se razonó en el apartado b), si se elimina una
  de las $N$ cargas, por ejemplo $q_{i}$, la resultante
  de las fuerzas repulsivas que las restantes cargas ejercen sobre $q$ sería
  $\vvv{R}' = \sum_{k=0}^{i-1} \vvv{F}_{k} + \sum_{k=i+1}^{N-1} \vvv{F}_{k}$
  \[
    \vvv{R}
    = \sum_{k=0}^{N-1} \vvv{F}_{k}
    = \sum_{k=0}^{i-1} \vvv{F}_{k} + \vvv{F}_{i} + \sum_{k=i+1}^{N-1} \vvv{F}_{k}
    = \vvv{F}_{i} + \vvv{R}'
    = 0
  \]
  Despejando la suma de las fuerzas que actúan sobre la carga de prueba
  demostramos que la nueva resultante es la opuesta de la fuerza que ejercería
  la carga que hemos eliminado, que en nuestro ejemplo es
  $-\vvv{F}_{i}$
  \[
    \vvv{R}' = \sum_{k=0}^{i-1} \vvv{F}_{k} + \sum_{k=i+1}^{N-1} \vvv{F}_{k}
    = -\vvv{F}_{i}
  \]
  
  

\end{soluc}

\end{document}


%%% Local Variables:
%%% coding: utf-8
%%% mode: latex
%%% TeX-engine: luatex
%%% TeX-master: t
%%% End:
