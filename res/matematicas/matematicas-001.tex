% matematicas-001.tex
%
% Copyright (C) 2019-2025 José A. Navarro Ramón <janr.devel@gmail.com>

% ....................................................................
% OBSERVACIÓN: Se puede dar formato al búffer en AUCTeX con: M-x L-buff RET
% ....................................................................

\documentclass[a4paper,10pt]{article}

\usepackage{../edclasica-res.pkg}
\usepackage{../edclasica-res.defs}

% ********************************************************************
% ******* DEFINICIONES DE ESTE EJERCICIO *****************************
% ********************************************************************
% Bloques de ejercicios
\renewcommand*{\mainsubject}{Análisis Vectorial}
\renewcommand*{\parte}{EJERCICIOS DE ELECTRODINÁMICA CLÁSICA}
\renewcommand*{\tipoBloque}{Problema}
\renewcommand*{\bloque}{1}
\renewcommand*{\hoja}{1}
\renewcommand*{\ejBloque}{1}
% Fuente: examen
% \renewcommand*{\ejExamen}{1}
\renewcommand*{\fuente}{Introduction to Electrodynamics - Griffiths - 4th Ed.
  Exercise 1.1. Page 4.}

% *****************************************************************************
\directlua{dofile "lua/cartesianas_esfericas.lua"}

\newcommand*{\xcoord}[3]{%
  \directlua{tex.write(
    spherical2x(#1,#2,#3))}
}

\newcommand*{\ycoord}[3]{%
  \directlua{tex.write(
    spherical2y(#1,#2,#3))}
}

\newcommand*{\zcoord}[3]{%
  \directlua{tex.write(
    spherical2z(#1,#2,#3))}
}

\begin{document}
% ############################ ENUNCIADO ######################################
% Entrada en el índice del fichero 'pdf' 'enunciado.0'
\pdfbookmark[0]{Enunciado}{enunciado}

% --------------------------------------------------------------------
% 
% --------------------------------------------------------------------
\begin{qboxshort}
  Demuestre que los productos escalar y vectorial son distributivos
usando las definiciones
  \[ \vvv{A}\cdot \vvv{B} \equiv A B \cos\theta
  \] y
  \[ \vvv{A}\prodvec \vvv{B} \equiv A B \sin\theta \,\xhat{n}
  \] y los diagramas que se necesiten
  \begin{subejercicio}
  \item Cuando los tres vectores son coplanares.
  \item En el caso general.
  \end{subejercicio}
\end{qboxshort}

% ######################### RESOLUCIÓN ###############################

% ================ APARTADO a) =======================================
% Entrada en el índice del fichero 'pdf'
\pdfbookmark[0]{Apartado a}{a}

\begin{soluc}

\item  
  Demostraremos las propiedades mediante tres vectores coplanares
  genéricos $\vvv{a}$, $\vvv{b}$ y $\vvv{c}$ que vemos representados
  a la izquierda de la figura~\ref{fig:diagrama}. Se han elegido
  estos vectores para que se pueda seguir fácilmente el razonamiento
  geométrico, pero valdría cualquier trio de vectores
  
  \begin{figure}[ht]
    \def\scl{1}
    \def\lngx{4.3}
    \def\lngy{2.8}
    \def\anga{60}
    \def\lnga{1.8}
    \def\angb{25}
    \def\lngb{1.5}
    \def\angc{0}
    \def\lngc{3.5}
    \begin{minipage}{0.3\linewidth}
    \begin{tikzpicture}[scale=\scl]
      % Coordenadas
      \coordinate (O) at (0,0);
      \coordinate (a) at (\anga:\lnga);
      \coordinate (b) at (\angb:\lngb);
      \coordinate (c) at (\angc:\lngc);
      % Eje x
      \draw[-{Latex}] (O) -- (\lngx,0) coordinate (ex);
      \node[below] at (ex) {$x$};
      % Eje y
      \draw[-{Latex}] (O) -- (0,\lngy) coordinate (ey);
      \node[left] at (ey) {$y$};
      % Vector a
      \draw[-{Latex[round]},ultra thick,] (O) -- (a)
      node[above] at (a) {$\vvv{a}$};
      % Vector b
      \draw[-{Latex[round]},ultra thick] (O) -- (b)
      node[above] {$\vvv{b}$};
      % Vector c
      \draw[-{Latex[round]},ultra thick] (O) -- (c);
      \node[below] at (c) {$\vvv{c}$};
    \end{tikzpicture}
  \end{minipage}
  \hspace{1em}
    \begin{minipage}{0.3\linewidth}
    \begin{tikzpicture}[scale=\scl]
      % Coordenadas
      \coordinate (O) at (0,0);
      \coordinate (a) at (\anga:\lnga);
      \coordinate (b) at (\angb:\lngb);
      \coordinate (c) at (\angc:\lngc);
      % Eje x
      \draw[-{Latex}] (O) -- (\lngx,0) coordinate (ex);
      \node[below] at (ex) {$x$};
      % Eje y
      \draw[-{Latex}] (O) -- (0,\lngy) coordinate (ey);
      \node[left] at (ey) {$y$};
      % Vector a
      \draw[-{Latex[round]},ultra thick,] (O) -- (a)
      node[above] at (a) {$\vvv{a}$};
      % Vector b
      \draw[-{Latex[round]},ultra thick] (a) -- +(b) coordinate (bpl)
      node[above] {$\vvv{b}$};
      % Vector c
      \draw[-{Latex[round]},ultra thick] (O) -- (c);
      \node[below] at (c) {$\vvv{c}$};
      % Líneas guía
      \draw[black!20,ultra thin] (a) -- (a |- ex);
      \draw[black!20,ultra thin] (bpl) -- (bpl |- ex);
      % Vector a+b
      \draw[-{Latex[round]},ultra thick,red,shorten >=1pt]
      (O) -- node[sloped,below] {$\vvv{a}+\vvv{b}$} (bpl);
      % Proyecciones
      \begin{scope}[line width=3pt]
        \draw[green!60!black]
        ($(O) - (0,3pt)$) -- ($(a |- ex)-(0,3pt)$) coordinate (ax);
        \draw[blue] (ax) -- ($(bpl |- ex)-(0,3pt)$) coordinate (abx);
        \draw[red] ($(O)-(0,7pt)$) -- ($(bpl |- ex)-(0,7pt)$);
      \end{scope}
    \end{tikzpicture}
  \end{minipage}
  \hspace{1em}
    \begin{minipage}{0.3\linewidth}
    \begin{tikzpicture}[scale=\scl]
      % Coordenadas
      \coordinate (O) at (0,0);
      \coordinate (a) at (\anga:\lnga);
      \coordinate (b) at (\angb:\lngb);
      \coordinate (c) at (\angc:\lngc);
      % Eje x
      \draw[-{Latex}] (O) -- (\lngx,0) coordinate (ex);
      \node[below] at (ex) {$x$};
      % Eje y
      \draw[-{Latex}] (O) -- (0,\lngy) coordinate (ey);
      \node[left] at (ey) {$y$};
      % Vector a
      \draw[-{Latex[round]},ultra thick,] (O) -- (a)
      node[above] at (a) {$\vvv{a}$};
      % Vector b
      \draw[-{Latex[round]},ultra thick] (a) -- +(b) coordinate (bpl)
      node[above] {$\vvv{b}$};
      % Vector c
      \draw[-{Latex[round]},ultra thick] (O) -- (c);
      \node[below] at (c) {$\vvv{c}$};
      % Líneas guía
      \draw[black!20,ultra thin] (a) -- (a -| ey);
      \draw[black!20,ultra thin] (bpl) -- (bpl -| ey);
      % Vector a+b
      \draw[-{Latex[round]},ultra thick,red,shorten >=1pt]
      (O) -- node[sloped,below] {$\vvv{a}+\vvv{b}$} (bpl);
      % Proyecciones
      \begin{scope}[line width=3pt]
        \draw[green!60!black]
           ($(O) - (2pt,0)$) -- ($(a -| ey)-(2pt,0)$) coordinate (ay);
        \draw[blue] (ay) -- ($(bpl -| ey)-(2pt,0)$) coordinate (aby);
        \draw[red] ($(O)-(6.5pt,0)$) -- ($(bpl -| ey)-(6.5pt,0)$);
      \end{scope}
    \end{tikzpicture}
  \end{minipage}
  
  \caption{En la figura de la izquierda tenemos tres vectores
    coplanares $\vvv{a}$, $\vvv{b}$ y $\vvv{c}$. En la figura
    de enmedio se representan sus proyecciones
    sobre el eje $x$. A la derecha se representan las proyecciones
    sobre el eje $y$.
  }
  \label{fig:diagrama}
  \end{figure}

  
% Entrada en el índice del fichero 'pdf'
\pdfbookmark[1]{Producto escalar en 2D}{a_esc_2D}

\bigskip
\textbf{Producto escalar en 2D}
  
  Según la definición, el producto escalar de dos vectores
  es igual al módulo de la proyección de uno de ellos sobre el otro,
  multiplicado por el módulo del otro, siendo $\alpha$ el ángulo que
  forman los vectores $\vvv{a}$ y $\vvv{c}$.
  \[
    \vvv{a}\cdot\vvv{c}
    =
    |\vvv{a}|\,|\vvv{c}|\,\cos\alpha
    = \left(|\vvv{a}|\,\cos\alpha\right)\,|\vvv{c}|
    = \text{Proy}_{\vvv{c}}(\vvv{a})\,|\vvv{c}|
  \]
  Téngase en cuenta que $\text{Proy}_{\vvv{c}} (\vvv{a})$ es un escalar.

  Si el producto escalar es distributivo, se debe cumplir que
  \begin{equation}\label{eq:escalardistrib}
    (\vvv{a} + \vvv{b}) \cdot \vvv{c}
    =
    \vvv{a}\cdot \vvv{c} + \vvv{b}\cdot \vvv{c}
  \end{equation}

  Analizamos el miembro de la izquierda de la
  ecuación~(\ref{eq:escalardistrib})
  \[
    (\vvv{a} + \vvv{b})\cdot \vvv{c}
    = |\vvv{a} + \vvv{b}|\,|\vvv{c}|\,\cos\theta
    = \text{Proy}_{\vvv{c}}(\vvv{a}+\vvv{b}) \,\,\|\vvv{c}|
  \]
  donde la proyección de la suma de los vectores $\vvv{a} + \vvv{b}$
  está representada en rojo en el centro de la figura.
  
  Ahora desarrollamos el miembro de la derecha
  \begin{align*}
    \vvv{a}\cdot\vvv{c} + \vvv{b}\cdot\vvv{c}
    &=
    |\vvv{a}|\,|\vvv{c}|\,\cos\alpha
    +
    |\vvv{b}|\,|\vvv{c}|\,\cos\beta
    =
    \text{Proy}_{\vvv{c}} (\vvv{a}) \,\,|\vvv{c}|
    +
    \text{Proy}_{\vvv{c}} (\vvv{b}) \,\,|\vvv{c}|
    \\&=
   (\text{Proy}_{\vvv{c}}(\vvv{a})+\text{Proy}_{\vvv{c}})\,\,|\vvv{c}|
   \\ &=
   (\text{Proy}_{\vvv{c}}(\vvv{a}) + \text{Proy}_{\vvv{c}}(\vvv{b}))
   \,|\vvv{c}|
  \end{align*}
  Las proyecciones de $\vvv{a}$ y de $\vvv{b}$ sobre $\vvv{c}$
  se representan en verde y en azul, respectivamente, en el
  centro de la figura.

  En la misma figura se ve que los dos miembros son iguales, por lo
  que se demuestra geométricamente que el producto escalar tiene
  la propiedad distributiva (al menos cuando los vectores implicados
  son coplanarios).

  % Entrada en el índice del fichero 'pdf'
  \pdfbookmark[1]{Producto vectorial en 2D}{a_vect_2D}
  
  \bigskip
  \textbf{Producto vectorial en 2D}
  
  Según la definición, el producto vectorial de dos vectores
  es igual a un vector perpendicular a ambos vectores indicado
  por el vector normal $\xhat{n}$, cuyo módulo
  es igual al producto de los módulos de los vectores, multiplicado
  por el seno del ángulo $\alpha$ formado por los vectores
  \[
    \vvv{a}\prodvec\vvv{c}
    =
    |\vvv{a}|\,|\vvv{c}|\,\sin\alpha\,\xhat{n}
  \]
    
  Si el producto vectorial es distributivo, se debe cumplir que
  \begin{equation}\label{eq:vectorialdistrib}
    (\vvv{a} + \vvv{b}) \prodvec \vvv{c}
    =
    \vvv{a}\prodvec\vvv{c} + \vvv{b}\prodvec\vvv{c}
  \end{equation}

  Analizamos el miembro de la izquierda de la
  ecuación~(\ref{eq:vectorialdistrib})
  \[
    (\vvv{a} + \vvv{b})\prodvec \vvv{c}
    = |\vvv{a} + \vvv{b}|\,|\vvv{c}|\,\sin\theta\,\xhat{n}
    = (|\vvv{a} + \vvv{b}|)\,\sin\theta\,\xhat{n}
  \]
  donde $(|\vvv{a} + \vvv{b}|)\,\sin\theta$ está representado en rojo
  a la derecha de la figura~\ref{fig:diagrama}.
  
  Ahora desarrollamos el miembro de la derecha de la igualdad
  \begin{align*}
    \vvv{a}\prodvec\vvv{c} + \vvv{b}\prodvec\vvv{c}
    &=
    |\vvv{a}|\,|\vvv{c}|\,\sin\alpha\,\xhat{n}
    +
      |\vvv{b}|\,|\vvv{c}|\,\sin\beta\,\xhat{n}
    =
    (|\vvv{a}|\,\sin\alpha + |\vvv{b}|\,\sin\beta)\,|\vvv{c}|\,\xhat{n}
  \end{align*}
  El primer sumando del paréntesis $|\vvv{a}|\,\sin\alpha$ se representa
  en verde a la derecha de la figura. El segundo sumando
  $|\vvv{a}|\,\cos\beta$ aparece en verde. En esa figura se aprecia
  que la suma de estos dos términos es igual al correspondiente del
  primer miembre, por lo que queda demostrada la propiedad distributiva
  del producto vectorial, al menos cuando los tres vectores son
  coplanarios.

% ================ APARTADO b) =======================================
% Entrada en el índice del fichero 'pdf'
\pdfbookmark[0]{Apartado b}{b}

\item
  En este apartado demostraremos la propiedad distributiva del producto escalar
  y del vectorial para vectores en tres dimensiones.

  % Entrada en el índice del fichero 'pdf'
  \pdfbookmark[1]{Producto escalar 3D}{b-prod-escalar-3D}

  
  % Entrada en el índice del fichero 'pdf'
  \pdfbookmark[2]{Conceptos básicos}{b-conceptos-basicos}
  
  \bigskip
  \textbf{Conceptos básicos}

   Antes de empezar las demostraciones vamos a demostrar algunas
  propiedades útiles de los vectores

  \begin{itemize}
  \item Cosenos directores
    
  \begin{figure}[ht]
  \begin{minipage}{.40\linewidth}
    \def\scl{4}
    %\tdplotsetmaincoords{70}{110}
    \tdplotsetmaincoords{70}{110}
    % 
    \pgfmathsetmacro{\rvec}{1.0}
    \pgfmathsetmacro{\thetavec}{30}
    \pgfmathsetmacro{\phivec}{50}
    %
    \pgfmathsetmacro{\xvec}{\xcoord{\rvec}{\thetavec}{\phivec}}
    \pgfmathsetmacro{\yvec}{\ycoord{1.0}{30}{50}}
    \pgfmathsetmacro{\zvec}{\zcoord{1.0}{30}{50}}

    \begin{tikzpicture}[scale=\scl,tdplot_main_coords,font=\normalsize]
      \coordinate (O) at (0,0,0);
      \draw[thick,->] (0,0,0) -- (1.3,0,0) node[anchor=north east]{$x$};
      \draw[thick,->] (0,0,0) -- (0,1,0) node[anchor=north west]{$y$};
      \draw[thick,->] (0,0,0) -- (0,0,1) node[anchor=south]{$z$};
      \tdplotsetcoord{P}{\rvec}{\thetavec}{\phivec}
      \tdplotsetthetaplanecoords{\phivec}
      \tdplotdrawarc[tdplot_rotated_coords]{(0,0,0)}{0.4}{0}%
      {\thetavec}{above right=0pt and -3pt}{$\gamma$}

      \tdplotdefinepoints(0,0,0)(\xvec,\yvec,\zvec)(1,0,0);
      \tdplotdrawpolytopearc[thick]{.3}{anchor=south east}{$\alpha$}

      \tdplotdefinepoints(0,0,0)(\xvec,\yvec,\zvec)(0,1,0);
      \tdplotdrawpolytopearc[thick]{.15}{anchor=south west}{$\beta$}

      \draw[-{stealth},color=red,ultra thick] (O) -- (P)
      node[above] {$\vvv{v}$};
      
    \end{tikzpicture}
  \caption{La suma de los cuadrados de los cosenos de los ángulos, $\alpha$,
    $\beta$ y $\gamma$, que forma un vector cualquiera con los ejes de
    coordenadas es la unidad.}
  \label{fig:cosenosdirectores}
\end{minipage}
\begin{minipage}{.60\linewidth}
  Un vector cualquiera $\vvv{v}$ se puede representar en el espacio euclídeo
  de tres dimensiones mediante sus tres componentes cartesianas, que son
  el producto de su módulo por el coseno del ángulo, $\alpha$, $\beta$ o
  $\gamma$, que forma el vector con cada uno de los ejes
  \begin{align*}
    \vvv{v}
    &= v_{x}\, \xhat{i} + v_{y}\, \xhat{j} + v_{z}\, \xhat{k}
    = |\vvv{v}|\cos\alpha\, \xhat{i}
    + |\vvv{v}|\cos\beta\, \xhat{j}
    + |\vvv{v}|\cos\gamma\, \xhat{k}\\
    &= |\vvv{v}|\,(\cos\alpha\,\xhat{i}+\cos\beta\,\xhat{j}+\cos\gamma\,\xhat{k})
  \end{align*}
  
    Pero un vector cualquiera $\vvv{v}$ se puede representar también como el producto de
    su módulo por el vector unidad $\vvv{u}_{\vvv{v}}$ con la misma dirección y sentido que el vector considerado
    \[
      \vvv{v} = |\vvv{v}|\,\xhat{u}_{\vvv{v}}
    \]

    De lo anterior deducimos que el vector unidad con la misma dirección y
    sentido que $\vvv{v}$ es
    \[
      \xhat{u}_{v}
      = \cos\alpha\,\xhat{i} + \cos\beta\,\xhat{j} + \cos\gamma\,\xhat{k}
    \]    
    
  \end{minipage}
\end{figure}
  
Como $\xhat{u}_{v}$ es un vector unidad, la suma de de los cuadrados
de sus componentes debe valer la unidad
    \[
       \cos^{2}\alpha + \cos^{2}\beta + \cos^{2}\gamma = 1
    \]

    Los cosenos de los ángulos que forma un vector $\vvv{v}$ con los ejes
    de coordenadas se denominan \emph{cosenos directores} y son las
    componentes cartesianas del vector unidad que tiene la misma
    dirección y sentido que $\vvv{v}$.

  \item Producto escalar de los vectores unidad en los ejes cartesianos

    Multiplicaremos entre sí los vectores unidad en los ejes de
    coordenadas cartesianas, $\xhat{\i}$, $\xhat{\j}$ y $\xhat{k}$.
    El producto de dos diferentes vectores unidad da cero, por ejemplo
    \[
      \xhat{\i}\cdot\xhat{\j}
      =
      |\xhat{\i}|\,|\xhat{\j}|\,cos(\pi/2) = 0
    \]

    En cambio el producto escalar de uno de ellos por sí mismo da
    la unidad, como
    \[
      \xhat{\i}\cdot\xhat{\i}
      =
      |\xhat{\i}|\,|\xhat{\i}|\,cos\ang{0} = 1
    \]

    Para resumir el resultado de estas multiplicaciones representaremos
    cada eje de coordenadas por el ordinal, $
    1,2,3$ y los vectores
    unidad correspondientes como $\xhat{u}_{1}$, $\xhat{u}_{2}$ y
    $\xhat{u}_{3}$. Así
    \[
      \xhat{u}_{k}\cdot\xhat{u}_{l}
      =
      \delta_{kl}
    \]
    donde la delta de Kronecker, $\delta_{kl}$ vale uno si los índices
    $k$ y $l$ son iguales y cero en caso contrario.

  \item Producto escalar en función de las componentes

    Multipliquemos escalarmente los vectores $\vvv{a}$ y $\vvv{b}$
    \begin{equation}\label{eq:prod-escalar-componentes}
      \vvv{a}\cdot\vvv{b}
      =
      \left(\sum_{k=1}^{3}a_{k}\,\xhat{u}_{k}\right)
      \cdot
      \left(\sum_{k=1}^{3}a_{k}\,\xhat{u}_{k}\right)
      =
      \sum_{k=1}^{3}\sum_{l=1}^{3} a_{k}\,b_{l}\,\xhat{u}_{k}\cdot\xhat{u}_{k}
      =
      \sum_{k=1}^{3}\sum_{l=1}^{3} a_{k}\,b_{l}\,\delta_{kl}
      =
      \sum_{k=1}^{3} a_{k} b_{k}
    \end{equation}
    Este método para calcular el producto escalar de dos vectores suele
    ser muy útil en algunas ocasiones.

  \item Coseno del ángulo formado por dos vectores

    Representaremos el ángulo que forma el vector $\vvv{a}$ con el eje
    $k$ por $\alpha_{ak}$ y el coseno director correspondiente por
    $\cos\alpha_{ak}$
    \[
      \vvv{a}\cdot\vvv{b}
      =
      \sum_{k=1}^{3} a_{k} b_{k}
      =
      \sum_{k=1}^{3} |\vvv{a}|\,\cos\alpha_{ak}|\vvv{b}|\,\cos\alpha_{bk}
      =
      |\vvv{a}|\,|\vvv{b}|\,\sum_{k=1}^{3} \cos\alpha_{ak}\,\cos\alpha_{bk}
      =
      |\vvv{a}|\,|\vvv{b}|\,\cos\alpha_{ab}
    \]

    De aquí se deduce que el coseno del ángulo que forman $\vvv{a}$ y
    $\vvv{b}$, $\alpha_{ab}$, es
    \begin{equation}\label{eq:angulo-dos-vectores}
      \cos\alpha_{ab}
      =
      \sum_{k=1}^{3} \cos\alpha_{ak}\,\cos\alpha_{bk}
    \end{equation}

  \end{itemize}

  Ahora demostraremos la propiedad distributiva del producto escalar,
  ecuación~(\ref{eq:escalardistrib}) de dos formas diferentes

  \[
    (\vvv{a}+\vvv{b})\cdot\vvv{c}
    =
    \vvv{a}\cdot\vvv{c} + \vvv{b}\cdot\vvv{c}
  \]

  
  % Entrada en el índice del fichero 'pdf'
  \pdfbookmark[2]{Producto escalar en 3D - I}{b-prod-escalar-I}
  
  \bigskip
  \textbf{Producto escalar en 3D - I}
 
  \[
    (\vvv{a} + \vvv{b})\cdot \vvv{c}
    =
    \left(\sum_{k=1}^{3} (a_{k} + b_{k})\,\xhat{u}_{k}\right)
    \cdot\sum_{l=1}^{3} c_{l}\,\xhat{u}_{l}
    =
    \sum_{k=1}^{3} (a_{k}+b_{k})\,c_{k}
    =
    \sum_{k=1}^{3} a_{k}\,c_{k}
    +
    \sum_{k=1}^{3} b_{k}\,c_{k}
    =
    \vvv{a}\cdot\vvv{c}
    +
    \vvv{b}\cdot\vvv{c}
  \]
  Se ha llegado a esta conclusión porque los $a_{k}$, $b_{k}$ y $c_{k}$ son números
  reales y la suma y el producto son las operaciones conocidas sobre números
  reales, cumplen la propiedad distributiva.

  % Entrada en el índice del fichero 'pdf'
  \pdfbookmark[2]{Producto escalar en 3D - II}{b-prod-escalar-II}
  
  \bigskip
  \textbf{Producto escalar en 3D - II}
  
  Supongamos tres vectores $\vvv{a}$, $\vvv{b}$ y $\vvv{c}$
  en el espacio, no nulos. Elegimos los ejes cartesianos $x$, $y$
  y $z$ de manera que uno de los ejes, por ejemplo el eje $x$,
  coincida con el vector $\vvv{c}$.
  Los otros dos vectores, $\vvv{a}$ y $\vvv{b}$, podrán tener cualquier
  dirección y sentido en el espacio.
  \begin{align*}
    \vvv{a}
    &=
    |\vvv{a}|\,
      (\cos\alpha_{1}\,\xhat{i}
      +\cos\beta_{1}\,\xhat{j}
      +\cos\gamma_{1}\,\xhat{k})\\
    \vvv{b}
    &=
    |\vvv{b}|\,
      (\cos\alpha_{2}\,\xhat{i}
      +\cos\beta_{2}\,\xhat{j}
      +\cos\gamma_{2}\,\xhat{k})\\
    \vvv{c}
    &=
      |\vvv{c}|\,\cos\alpha_{3}\,\xhat{i}
  \end{align*}

 \begin{align*}
    \vvv{a}+\vvv{b}
    &=
      |\vvv{a}|
      (\cos\alpha_{1}\,\xhat{i}
      +\cos\beta_{1}\,\xhat{j}
      +\cos\gamma_{1}\,\xhat{k})
      +
      |\vvv{b}|\,
      (\cos\alpha_{2}\,\xhat{i}
      +\cos\beta_{2}\,\xhat{j}
      +\cos\gamma_{2}\,\xhat{k})\\
    &=
      (|\vvv{a}|\,\cos\alpha_{1} + |\vvv{b}|\,\cos\alpha_{2})\,\xhat{i}
      +
      (|\vvv{a}|\,\cos\beta_{1} + |\vvv{b}|\,\cos\beta_{2})\,\xhat{j}
      +
      (|\vvv{a}|\,\cos\gamma_{1} + |\vvv{b}|\,\cos\gamma_{2})\,\xhat{k}\\
    &=
      |\vvv{a}+\vvv{b}|\,\left[
      \left(
        \frac{|\vvv{a}|\,\cos\alpha_{1} + |\vvv{b}|\,\cos\alpha_{2}}
             {|\vvv{a}+\vvv{b}|}
      \right)\,\xhat{i}
      +
      \left(
        \frac{|\vvv{a}|\,\cos\beta_{1} + |\vvv{b}|\,\cos\beta_{2}}
             {|\vvv{a}+\vvv{b}|}
      \right)\,\xhat{j}
      +
      \left(
        \frac{|\vvv{a}|\,\cos\gamma_{1} + |\vvv{b}|\,\cos\gamma_{2}}
             {|\vvv{a}+\vvv{b}|}
      \right)\,\xhat{k}
      \right]\\
    &=
      |\vvv{a}+\vvv{b}|\,
      (\cos\alpha_{s}\,\xhat{i}
      +
      \cos\beta_{s}\,\xhat{j}
      +
      \cos\gamma_{s}\,\xhat{k})
 \end{align*}
 donde $\cos\alpha_{s}$, $\cos\beta_{s}$ y $\cos\gamma_{s}$ son los
 cosenos directores de los ángulos del vector suma $\vvv{a}+\vvv{b}$
 con los ejes de coordenadas.
 
 Calculamos por separado cada término teniendo en cuenta que el
 ángulo que forma $\vvv{a}$ con $\vvv{c}$ es el ángulo que forma
 con el eje $x$, $\alpha_{1}$ y $\vvv{b}$ forma un ángulo $\alpha_{2}$
 con $\vvv{c}$
 \begin{align*}
   \vvv{a}\cdot\vvv{c}
   &=
   |\vvv{a}|\,|\vvv{c}|\,\cos\alpha_{1}\\
   \vvv{b}\cdot\vvv{c}
   &=
   |\vvv{b}|\,|\vvv{c}|\,\cos\alpha_{2}
 \end{align*}

 El primer término de la propiedad distributiva es
 \begin{align*}
    (\vvv{a}+\vvv{b})\cdot\vvv{c}
    &=
      \cancelout{|\vvv{a}+\vvv{b}|}\,|\vvv{c}|\,
      \frac{|\vvv{a}|\,\cos\alpha_{1}+|\vvv{b}|\,\cos\alpha_{2}}
      {\cancelout{|\vvv{a} + \vvv{b}|}}
    =
      |\vvv{c}|\,(|\vvv{a}|\,\cos\alpha_{1}+|\vvv{b}|\,\cos\alpha_{2})  
  \end{align*}

  El segundo término es igual al primero
  \[
    \vvv{a}\cdot\vvv{c} + \vvv{b}\cdot\vvv{c}
    =
    |\vvv{a}|\,|\vvv{c}|\,\cos\alpha_{1}+|\vvv{b}|\,|\vvv{c}|\,\cos\alpha_{2}
    =
    |\vvv{c}|\,(|\vvv{a}|\,\cos\alpha_{1}+|\vvv{b}\,\cos\alpha_{2})
  \]

  Queda demostrada la propiedad distributiva para el producto escalar.

  % Entrada en el índice del fichero 'pdf'
  \pdfbookmark[1]{Producto vectorial 3D}{b-prod-vect-3D}
  
  \bigskip
  \textbf{Producto vectorial 3D}
  
  Ahora demostraremos la propiedad distributiva del producto vectorial,
  ecuación~(\ref{eq:vectorialdistrib})

  \[
    (\vvv{a}+\vvv{b})\prodvec\vvv{c}
    =
    \vvv{a}\prodvec\vvv{c} + \vvv{b}\prodvec\vvv{c}
  \]

  % Entrada en el índice del fichero 'pdf'
  \pdfbookmark[2]{Conceptos básicos}{b-conceptos-basicos-vect}
  
  \bigskip
  \textbf{Conceptos básicos}

  \begin{itemize}
  \item Definición de producto vectorial

    El producto vectorial $\vvv{a}\prodvec\vvv{b}$ se puede definir mediante
    las siguientes proposiciones
    \begin{enumerate}
    \item El producto vectorial de dos vectores $\vvv{a}$ y $\vvv{b}$ es un vector
      perpendicular a ambos vectores.
    \item Hay dos posibles sentidos para el vector perpendicular. Se elige
      el sentido dado por la mano derecha
      \begin{quote}
        Se hace coincidir el vector $\vvv{a}$ con el índice,
        el vector $\vvv{b}$ con el dedo medio. El sentido
        del vector producto vectorial será el que indique el pulgar.
      \end{quote}
    \item El módulo del producto vectorial es el producto de los módulos de los
      dos vectores multiplicado por el seno del ángulo agudo o llano formado por
      los vectores
      \[
        \vvv{a}\prodvec\vvv{b}
        =
        |\vvv{a}|\,|\vvv{b}|\,\sin\alpha_{ab}
      \]
    \end{enumerate}

  \item El módulo del producto vectorial es igual al área del paralelepípedo
    formado por los vectores
    \begin{figure}[ht]
      \begin{minipage}{.45\linewidth}
      \def\scl{1}
      \def\alen{4}
      \def\aang{0}
      \def\blen{3}
      \def\bang{40}
      \def\vectwidth{1.6pt}
      \begin{tikzpicture}[scale=\scl]
        % Coordenadas más relevantes O, A, B, h, O'
        \coordinate (O) at (0,0);
        \coordinate (A) at (\aang:\alen);
        \coordinate (B) at (\bang:\blen);
        \coordinate (h) at  (B |- A);
        \path (B) -- ++(\aang:\alen) coordinate (op);

        % Relleno de triángulo
        \filldraw[fill=black!15] (O) -- (A) -- (B) -- cycle;
        
        % Líneas auxiliares
        % B -- O' -- A -- B
        \draw[black!50] (B) -- (op) -- (A) -- cycle;
        % altura
        \draw[red!50,ultra thin] (B) -- node[right,black] {$h$} (h);
        
        % Etiquetas de vértices
        \node[below left] at (O) {$O$};
        \node[below] at (A) {$A$};
        \node[above] at (B) {$B$};
        \node[right] at (op) {$O'$};
        \node[below] at (h) {$H$};

        % Vectores
        \draw[-{Latex[round]},line width=\vectwidth,name path=line a] (O) -- (A);
        \draw[-{Latex[round]},line width=\vectwidth,name path=line b] (O) -- (B);

      \end{tikzpicture}
      \caption{Construcción para demostrar que el módulo del producto
        vectorial de dos vectores es igual al área del paralelepípedo
        generado por los vectores.}
      \label{fig:area-paralelepipedo}
    \end{minipage}
    \hspace{1em}
    \begin{minipage}{.50\linewidth}
      El módulo del producto vectorial de los vectores $\vvv{a}$ y $\vvv{b}$,
      representados por los segmentos dirigidos $OA$ y $OB$, vale
      \begin{align*}
        |\vvv{a}\prodvec\vvv{b}|
        &=
        |\vvv{a}|\,|\vvv{b}|\,\sin\alpha
        =
        OA\,h
        =
        2\,\frac{OA\,h}{2}\\
        &=
        2\,\text{área del triángulo } AOB\\
        &=
        \text{área del parelelepípedo } OAO'B
      \end{align*}

      Donde $|\vvv{b}|\,\sin\alpha$ es el la altura $h$ del triángulo $AOB$ y
      $OA$ es la base del triángulo.
    \end{minipage}
    \end{figure}

    \item El producto vectorial expresado como determinante

    Sabemos por la proposición 1 en la definición del producto vectorial
    que el vector resultante $\vvv{c}=\vvv{a}\prodvec\vvv{b}$ del producto
    vectorial de dos vectores $\vvv{a}$ y $\vvv{b}$ es perpendicular a ambos.
    Por tanto el producto escalar de estos vectores es cero
    \begin{align*}
      \vvv{a}\cdot\vvv{c} &= 0\\
      \vvv{b}\cdot\vvv{c} &= 0
    \end{align*}

    Desarrollando el producto escalar por componentes porque estamos interesados
    en las componentes $c_{1}$, $c_{2}$ y $c_{3}$
    \begin{align*}
      a_{1} c_{1} + a_{2} c_{2} + a_{3} c_{3} &= 0\\
      b_{1} c_{1} + b_{2} c_{2} + b_{3} c_{3} &= 0
    \end{align*}

    Tenemos dos ecuaciones con tres incógnitas, por lo que tendremos que
    añadir más adelante otra condición para obtener las componentes del producto
    vectorial.

    Obtenemos las siguientes relaciones y llamaremos $K$ a la proporción
    \[
      \frac{c_{1}}{a_{2}b_{3}-a_{3}b_{2}}
      =
      \frac{c_{2}}{a_{3}b_{1}-a_{1}b_{3}}
      =
      \frac{c_{3}}{a_{1}b_{2}-a_{2}b_{1}}
      =
      K
    \]

    Despejamos las componentes del producto vectorial en función de $K$
    \begin{align*}
      c_{1} &= K(a_{2}b_{3} - a_{3}b_{2})\\
      c_{2} &= K(a_{3}b_{1} - a_{1}b_{3})\\
      c_{3} &= K(a_{1}b_{2} - a_{2}b_{1})
    \end{align*}

    El cuadrado del producto vectorial es $c^{2} = c_{1}^{2} + c_{2}^{2} + c_{3}^{2}$
    \begin{align*}
      c^{2}
      &=
        K^{2}\left[
          (a_{2}b_{3})^{2} + (a_{3}b_{1}-a_{1}b_{3})^{2} + (a_{1}b_{2}-a_{2}b_{1})^{2}
        \right]\\
      &=
        K^{2}\left[
        a_{2}^{2}b_{3}^{2} + a_{3}^{2}b_{2}^{2} - 2a_{2}a_{3}b_{2}b_{3}
        + a_{3}^{2}b_{1}^{2}+a_{1}^{2}b_{3}^{2} - 2a_{1}a_{3}b_{1}b_{3}
        + a_{1}^{2}b_{2}^{2}+a_{2}^{2}b_{1}^{2} - 2a_{1}a_{2}b_{1}b_{2}
        \right]\\
      &=K^{2}\left[
        a_{1}^{2}\left(b_{2}^{2} + b_{3}^{2}\right)
        + a_{2}^{2}\left(b_{1}^{2} + b_{3}^{2}\right)
        + a_{3}^{2}\left(b_{1}^{2} + b_{2}^{2}\right)
        - 2 (a_{1}a_{2}b_{1}b_{2} + a_{2}a_{3}b_{2}b_{3} + a_{1}a_{3}b_{1}b_{3})
        \right]
    \end{align*}

    Como $b^{2} = b_{1}^{2} + b_{2}^{2} + b_{3}^{2}$
    \begin{align*}
      b_{2}^{2} + b_{3}^{2} &= b^{2}-b_{1}^{2}\\
      b_{1}^{2} + b_{3}^{2} &= b^{2}-b_{2}^{2}\\
      b_{1}^{2} + b_{2}^{2} &= b^{2}-b_{3}^{2}\\
    \end{align*}

    Sustituimos en la expresión del cuadrado del producto vectorial
    \begin{align*}
      c^{2}
      &=
        K^{2}\left[
        a_{1}^{2}\left(b^{2} - b_{1}^{2}\right)
        + a_{2}^{2}\left(b^{2} - b_{2}^{2}\right)
        + a_{3}^{2}\left(b^{2} - b_{3}^{2}\right)
        - 2 (a_{1}a_{2}b_{1}b_{2} + a_{2}a_{3}b_{2}b_{3} + a_{1}a_{3}b_{1}b_{3})
        \right]\\
      &=
        K^{2}\left[
        b^{2}\left(a_{1}^{2} + a_{2}^{2} + a_{3}^{2}\right)
        - a_{1}^{2}b_{1}^{2} - a_{2}^{2}b_{2}^{2} - a_{3}^{2}b_{3}^{2}
        - 2 (a_{1}a_{2}b_{1}b_{2} + a_{2}a_{3}b_{2}b_{3} + a_{1}a_{3}b_{1}b_{3})
        \right]\\
      &=
        K^{2}\left[
          b^{2}a^{2} - (a_{1}b_{1} + a_{2}b_{2} + a_{3}b_{3})^{2}
        \right]
        =
        K^{2}\left[
          a^{2}b^{2} - (a b \cos\alpha_{ab})^{2}
        \right]\\
      &=
        K^{2}\left[
          a^{2}b^{2} - a^{2}b^{2}\cos^{2}\alpha_{ab}
        \right]
        =
        K^{2} a^{2} b^{2} \left(1-\cos^{2}\alpha_{ab}\right)\\
      &=
        K^{2} a^{2} b^{2} \sin^{2}\alpha_{ab}
    \end{align*}

    Entonces
    \[
      c = \pm |K| a b \sin\alpha_{ab}
    \]

    Según la definición de producto vectorial deducimos que $|K| = 1$.
    Sólo nos queda determinar el signo de $K$. Para obtenerlo nos fijamos
    en el segundo punto de la definición de producto vectorial.

    Calcularemos un producto vectorial sencillo, como
    $\xhat{u}_{1}\prodvec\xhat{u}_{2}$ para aplicar la regla de la mano derecha
    y determinar el signo de $K$. Aplicando dicha regla obtenemos
    \[
      \xhat{u}_{1}\prodvec\xhat{u}_{2}
      =
      1\cdot 1\sin\ang{90}\,\xhat{u}_{3} = 1\,\xhat{u}_{3} = \xhat{u}_{3}
    \]
    
    \begin{align*}
      \xhat{u}_{1} &= (a_{1}, a_{2}, a_{3}) = (1, 0, 0)\\
      \xhat{u}_{2} &= (b_{1}, b_{2}, b_{3}) = (0, 1, 0)\\
      \xhat{u}_{3} &= (c_{1}, c_{2}, c_{3}) = (0, 0, 1)
    \end{align*}

    Para que se obtenga el resultado con el producto vectorial sencillo debemos
    hacer que $K$ sea positivo ($K = +1$)
    \begin{align*}
      c_{1} &= 1\,(0\cdot 0 - 0\cdot 1) = 0\\
      c_{2} &= 1\,(0\cdot 0 - 1\cdot 0) = 0\\
      c_{3} &= 1\,(1\cdot 1 - 0\cdot 0) = 1
    \end{align*}

    Así, las componentes serán
    \begin{align*}
      c_{1} &= a_{2}b_{3} - a_{3}b_{2}\\
      c_{2} &= a_{3}b_{1} - a_{1}b_{3}\\
      c_{3} &= a_{1}b_{2} - a_{2}b_{1}
    \end{align*}

    Estas componentes se pueden obtener representando el producto vectorial
    como un determinante
    \begin{equation}\label{eq:prod-vectorial-determinante}
      \vvv{a}\prodvec\vvv{b}
      =
      \begin{vmatrix}
        \xhat{u}_{1} & \xhat{u}_{2} & \xhat{u}_{3}\\
        a_{1} & a_{2} & a_{3}\\
        b_{1} & b_{2} & b_{3}
      \end{vmatrix}
    \end{equation}
  \end{itemize}

  % Entrada en el índice del fichero 'pdf'
  \pdfbookmark[2]{Producto vectorial en 3D}{b-prod-vectorial}
  
  \bigskip
  \textbf{Producto vectorial en 3D}
  
  Demostraremos la propiedad distributiva aprovechando la linealidad de los
  determinantes
  \begin{align*}
    (\vvv{a}+\vvv{b})\prodvec\vvv{c}
    &=
      \begin{vmatrix}
        \xhat{u}_{1} & \xhat{u}_{2} & \xhat{u}_{3}\\
        a_{1} + b_{1} & a_{2} + b_{2} & a_{3} + b_{3}\\
        c_{1} & c_{2} & c_{3}
      \end{vmatrix}
    =
      \begin{vmatrix}
        \xhat{u}_{1} & \xhat{u}_{2} & \xhat{u}_{3}\\
        a_{1} & a_{2} & a_{3}\\
        c_{1} & c_{2} & c_{3}
      \end{vmatrix}
    +
      \begin{vmatrix}
        \xhat{u}_{1} & \xhat{u}_{2} & \xhat{u}_{3}\\
        b_{1} & b_{2} & b_{3}\\
        c_{1} & c_{2} & c_{3}
      \end{vmatrix}
    =
      \vvv{a}\prodvec\vvv{c} + \vvv{b}\prodvec\vvv{c}                    
  \end{align*}
  
\end{soluc}

\end{document}


%%% Local Variables:
%%% coding: utf-8
%%% mode: latex
%%% TeX-engine: luatex
%%% TeX-master: t
%%% End:
