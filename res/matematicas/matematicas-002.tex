% matematicas-002.tex
%
% Copyright (C) 2019-2025 José A. Navarro Ramón <janr.devel@gmail.com>
% ....................................................................
% OBSERVACIÓN: Se puede dar formato al búffer en AUCTeX con: M-x L-buff RET
% ....................................................................

\documentclass[a4paper,10pt]{article}

\usepackage{../edclasica-res.pkg}
\usepackage{../edclasica-res.defs}

% ********************************************************************
% ******* DEFINICIONES DE ESTE EJERCICIO *****************************
% ********************************************************************
% Bloques de ejercicios
\renewcommand*{\mainsubject}{Análisis Vectorial}
\renewcommand*{\parte}{EJERCICIOS DE ELECTRODINÁMICA CLÁSICA}
\renewcommand*{\tipoBloque}{Problema}
\renewcommand*{\bloque}{1}
\renewcommand*{\hoja}{1}
\renewcommand*{\ejBloque}{2}
% Fuente: examen
% \renewcommand*{\ejExamen}{1}
\renewcommand*{\fuente}{Introduction to Electrodynamics - Griffiths - 4th Ed.
  Exercise 1.2 Page 4.}

% *****************************************************************************
\directlua{dofile "lua/cartesianas_esfericas.lua"}

\newcommand*{\xcoord}[3]{%
  \directlua{tex.write(
    spherical2x(#1,#2,#3))}
}

\newcommand*{\ycoord}[3]{%
  \directlua{tex.write(
    spherical2y(#1,#2,#3))}
}

\newcommand*{\zcoord}[3]{%
  \directlua{tex.write(
    spherical2z(#1,#2,#3))}
}

\begin{document}
% ############################ ENUNCIADO ######################################
% Entrada en el índice del fichero 'pdf' 'enunciado.0'
\pdfbookmark[0]{Enunciado}{enunciado}

% --------------------------------------------------------------------
% 
% --------------------------------------------------------------------
\begin{qboxshort}
  ¿Es asociativo el producto vectorial?

  \[
    (\vvv{A}\prodvec\vvv{B})\prodvec\vvv{C}
    \questeq
    \vvv{A}\prodvec (\vvv{B}\prodvec\vvv{C})
  \]

  Demuéstrelo si lo es o muestre el contraejemplo, lo más simple posible,
  en caso contrario.
\end{qboxshort}

% ######################### RESOLUCIÓN ###############################

% ================ APARTADO a) =======================================
% Entrada en el índice del fichero 'pdf'
\pdfbookmark[0]{Solución}{sol}

\begin{soluc}

\item
  Demostraremos que el producto vectorial no tiene la
  propiedad asociativa. 

  Realizamos cálculos intermedios
  \begin{align*}
    \vvv{A}\prodvec\vvv{B}
    &=
    \begin{vmatrix}
      \xhat{u}_{x} & \xhat{u}_{y} & \xhat{u}_{z}\\
      A_{x} & A_{y} & A_{z}\\
      B_{x} & B_{y} & B_{z}
    \end{vmatrix}
    =
    \xhat{u}_{x}
    \begin{vmatrix}
      A_{y} & A_{z}\\
      B_{y} & B_{z}
    \end{vmatrix}
    +
    \xhat{u}_{y}
    \begin{vmatrix}
      A_{z} & A_{x}\\
      B_{z} & B_{x}
    \end{vmatrix}
    +
    \xhat{u}_{z}
    \begin{vmatrix}
      A_{x} & A_{y}\\
      B_{x} & B_{y}
    \end{vmatrix}\\
    &=
    (A_{y}B_{z}-A_{z}B_{y})\,\xhat{u}_{x}
    +
    (A_{z}B_{x}-A_{x}B_{z})\,\xhat{u}_{y}
    +
    (A_{x}B_{y}-A_{y}B_{x})\,\xhat{u}_{z}
  \end{align*}

    \begin{align*}
    \vvv{B}\prodvec\vvv{C}
    &=
    \begin{vmatrix}
      \xhat{u}_{x} & \xhat{u}_{y} & \xhat{u}_{z}\\
      B_{x} & B_{y} & B_{z}\\
      C_{x} & C_{y} & C_{z}
    \end{vmatrix}
    =
    \xhat{u}_{x}
    \begin{vmatrix}
      B_{y} & B_{z}\\
      C_{y} & C_{z}
    \end{vmatrix}
    +
    \xhat{u}_{y}
    \begin{vmatrix}
      B_{z} & B_{x}\\
      C_{z} & C_{x}
    \end{vmatrix}
    +
    \xhat{u}_{z}
    \begin{vmatrix}
      B_{x} & B_{y}\\
      C_{x} & C_{y}
    \end{vmatrix}\\
    &=
    (B_{y}C_{z}-B_{z}C_{y})\,\xhat{u}_{x}
    +
    (B_{z}C_{x}-B_{x}C_{z})\,\xhat{u}_{y}
    +
    (B_{x}C_{y}-B_{y}C_{x})\,\xhat{u}_{z}
  \end{align*}

  Ahora calculamos la expresión de la izquierda de la propiedad asociativa
  del enunciado
  \begin{align*}
    (\vvv{A}\prodvec\vvv{B})\prodvec\vvv{C}
    &=
    \begin{vmatrix}
      \xhat{u}_{x} & \xhat{u}_{y} & \xhat{u}_{z}\\
      A_{y}B_{z}-A_{z}B_{y} & A_{z}B_{x}-A_{x}B_{z} & A_{x}B_{y}-A_{y}B_{x}\\
      C_{x} & C_{y} & C_{z}
    \end{vmatrix}\\
    &=
      \xhat{u}_{x}
      {\small\begin{vmatrix}
        A_{z}B_{x}-A_{x}B_{z} & A_{x}B_{y}-A_{y}B_{x}\\
        C_{y} & C_{z}
      \end{vmatrix}}
      +
      \xhat{u}_{y}
      {\small\begin{vmatrix}
        A_{x}B_{y}-A_{y}B_{x} & A_{y}B_{z}-A_{z}B_{y}\\
        C_{z} & C_{x}
      \end{vmatrix}}
      +
      \xhat{u}_{z}
      {\small\begin{vmatrix}
        A_{y}B_{z}-A_{z}B_{y} & A_{z}B_{x}-A_{x}B_{z}\\
        C_{x} & C_{y}
      \end{vmatrix}}\\
    &=
      \left(A_{z}B_{x}C_{z}-A_{x}B_{z}C_{z}-A_{x}B_{y}C_{y}+A_{y}B_{x}C_{y}\right)\,\xhat{u}_{x}\\
    &\hspace{1em}+
      \left((A_{x}B_{y}C_{x}-A_{y}B_{x}C_{x}-A_{y}B_{z}C_{z}+A_{z}B_{y}C_{z}\right)\,\xhat{u}_{y}\\
    &\hspace{1em}+
      \left(A_{y}B_{z}C_{y}-A_{z}B_{y}C_{y}-A_{z}B_{x}C_{x}+A_{x}B_{z}C_{x}\right)\,\xhat{u}_{z}
  \end{align*}

  El segundo miembro es 
  \begin{align*}
    \vvv{A}\prodvec(\vvv{B}\prodvec\vvv{C})
    &=
    \begin{vmatrix}
      \xhat{u}_{x} & \xhat{u}_{y} & \xhat{u}_{z}\\
      A_{x} & A_{y} & A_{z}\\
      B_{y}C_{z}-B_{z}C_{y} & B_{z}C_{x}-B_{x}C_{z} & B_{x}C_{y}-B_{y}C_{x}
    \end{vmatrix}\\
    &=
      \xhat{u}_{x}
      {\small\begin{vmatrix}
        A_{y} & A_{z}\\
        B_{z}C_{x}-B_{x}C_{z} & B_{x}C_{y}-B_{y}C_{x}
      \end{vmatrix}}
      +
      \xhat{u}_{y}
      {\small\begin{vmatrix}
        A_{z} & A_{x}\\                              
        B_{x}C_{y}-B_{y}C_{x} & B_{y}C_{z}-B_{z}C_{y}
      \end{vmatrix}}
      +
      \xhat{u}_{z}
      {\small\begin{vmatrix}
        A_{x} & A_{y}\\                            
        B_{y}C_{z}-B_{z}C_{y} & B_{z}C_{x}-B_{x}C_{z}
      \end{vmatrix}}\\
    &=
      \left(A_{y}B_{x}C_{y}-A_{y}B_{y}C_{x}-A_{z}B_{z}C_{x}+A_{z}B_{x}C_{z}\right)\,\xhat{u}_{x}\\
    &\hspace{1em}+
      \left((A_{z}B_{y}C_{z}-A_{z}B_{z}C_{y}-A_{x}B_{x}C_{y}+A_{x}B_{y}C_{x}\right)\,\xhat{u}_{y}\\
    &\hspace{1em}+
      \left(A_{x}B_{z}C_{x}-A_{x}B_{x}C_{z}-A_{y}B_{y}C_{z}+A_{y}B_{z}C_{y}\right)\,\xhat{u}_{z}
  \end{align*}

  Para que se cumpla la propiedad asociativa, las dos expresiones deben
  ser iguales componente a componente, para todo conjunto de vectores
  $\vvv{A}$, $\vvv{B}$ y $\vvv{C}$
  \begin{itemize}
  \item Componente $x$
    \[
    \cancelout{A_{z}B_{x}C_{z}} - A_{x}B_{z}C_{z} - A_{x}B_{y}C_{y} + \cancelout{A_{y}B_{x}C_{y}}
    =
    \cancelout{A_{y}B_{x}C_{y}}-A_{y}B_{y}C_{x}-A_{z}B_{z}C_{x}+\cancelout{A_{z}B_{x}C_{z}}
  \]
  \begin{equation}\label{eq:componentex}
    A_{x}B_{z}C_{z} + A_{x}B_{y}C_{y}
    =
    A_{y}B_{y}C_{x} + A_{z}B_{z}C_{x}
    ~;\hspace{2em}
    \forall \vvv{A}, \vvv{B}, \vvv{C} \in \mathbb{R}^{3}
  \end{equation}

\item Componente $y$
    \[
      \cancelout{A_{x}B_{y}C_{x}}-A_{y}B_{x}C_{x}-A_{y}B_{z}C_{z}+\cancelout{A_{z}B_{y}C_{z}}
    =
      \cancelout{A_{z}B_{y}C_{z}}-A_{z}B_{z}C_{y}-A_{x}B_{x}C_{y}+\cancelout{A_{x}B_{y}C_{x}}
  \]
  \begin{equation}\label{eq:componentey}
      A_{y}B_{x}C_{x} + A_{y}B_{z}C_{z}
    =
      A_{z}B_{z}C_{y} + A_{x}B_{x}C_{y}
    ~;\hspace{2em}
    \forall \vvv{A}, \vvv{B}, \vvv{C} \in \mathbb{R}^{3}
  \end{equation}

\item Componente $z$
  \[
      \cancelout{A_{y}B_{z}C_{y}}-A_{z}B_{y}C_{y}-A_{z}B_{x}C_{x}+\cancelout{A_{x}B_{z}C_{x}}
      =
      \cancelout{A_{x}B_{z}C_{x}}-A_{x}B_{x}C_{z}-A_{y}B_{y}C_{z}+\cancelout{A_{y}B_{z}C_{y}}
  \]
  \begin{equation}\label{eq:componentez}
      A_{z}B_{y}C_{y} + A_{z}B_{x}C_{x}
      =
      A_{x}B_{x}C_{z} + A_{y}B_{y}C_{z}
    ~;\hspace{2em}
    \forall \vvv{A}, \vvv{B}, \vvv{C} \in \mathbb{R}^{3}
  \end{equation}

\end{itemize}

Las igualdades~(\ref{eq:componentex}), ~(\ref{eq:componentey})
y (\ref{eq:componentez}) no tienen que cumplirse necesariamente
para todo conjunto de vectores $\vvv{A}$, $\vvv{B}$ y $\vvv{C}$
(aunque en algunos casos particulares podrían cumplirse, como
cuando $\vvv{A} = \xhat{u}_{x}$, $\vvv{B} = \xhat{u}_{y}$ y
$\vvv{C} = \xhat{u}_{z}$),
por lo que concluimos que el producto vectorial no es asociativo.

Un contraejemplo sencillo podría ser,
$\vvv{A} = \xhat{u}_{x}$, $\vvv{B} = \xhat{u}_{y}$ y
$\vvv{C} = \xhat{u}_{y}$

Así
\begin{align*}
  (\vvv{A}\prodvec\vvv{B})\prodvec\vvv{C}
  =
  (\xhat{u}_{x}\prodvec\vvv{u}_{y})\prodvec\xhat{u}_{y}
  = \xhat{u}_{z}\prodvec\xhat{u}_{y}
  = -\xhat{u}_{x}
\end{align*}
que es diferente de
\begin{align*}
  \vvv{A}\prodvec(\vvv{B}\prodvec\vvv{C})
  =
  \xhat{u}_{x}\prodvec(\xhat{u}_{y}\prodvec\xhat{u}_{y})
  = \xhat{u}_{x}\prodvec\vvv{0}
  = \vvv{0}
\end{align*}


\end{soluc}

\end{document}


%%% Local Variables:
%%% coding: utf-8
%%% mode: latex
%%% TeX-engine: luatex
%%% TeX-master: t
%%% End:
