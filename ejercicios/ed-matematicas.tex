% ed-matematicas.tex
%
% Copyright (C) 2019-2025 José A. Navarro Ramón <janr.devel@gmail.com>
%
% ---------------------------------------------------------------------------
% ---------------------------------------------------------------------------
% HOJA
% ---------------------------------------------------------------------------
% ---------------------------------------------------------------------------
%\phantomsection
%\addcontentsline{toc}{subsection}{Hoja 1}
\setcounter{isubsheet}{1}

% Añade 'Contenidos' al índice pdf.
% \bookmark[level=2,dest=section]{Hoja 1}
% Nombre de enlace: 'cc1.2'
\pdfbookmark[2]{Hoja 1}{mathj01}

\begin{ejercicio}
% --------------------------------------------------------------------------
% Producto escalar y producto vectorial
% --------------------------------------------------------------------------
\item 
  Demuestre que los productos escalar y vectorial son distributivos
usando las definiciones
  \[ \vvv{A}\cdot \vvv{B} \equiv A B \cos\theta
  \] y
  \[ \vvv{A}\prodvec \vvv{B} \equiv A B \sin\theta \xhat{n}
  \] y los diagramas que se necesiten
  \begin{subejercicio}
  \item Cuando los tres vectores son coplanares.
  \item En el caso general.
  \end{subejercicio}

% ...........................................................................
  \showSolved{res/matematicas}{matematicas-001.pdf}
% ...........................................................................
\medskip
{\color{gray}
\hrule
}

% --------------------------------------------------------------------------
% Producto vectorial
% --------------------------------------------------------------------------
\item
  ¿Es asociativo el producto vectorial?
  \[
    (\vvv{A}\prodvec\vvv{B})\prodvec\vvv{C}
    \questeq
    \vvv{A}\prodvec (\vvv{B}\prodvec\vvv{C})
  \]
  Demuéstrelo si lo es o muestre el contraejemplo, lo más simple posible,
  en caso contrario.

% ...........................................................................
  \showSolved{res/matematicas}{matematicas-002.pdf}
% ...........................................................................
\medskip
{\color{gray}
\hrule
}

% --------------------------------------------------------------------------
%  Producto escalar
% --------------------------------------------------------------------------
\item
  Calcule el ángulo entre las diagonales principales de un cubo.

  % ...........................................................................
  {\footnotesize \textcolor{gray}{[No resuelto]}}
  %\showSolved{res/matematicas}{matematicas-003.pdf}
% ...........................................................................
\medskip
{\color{gray}
\hrule
}

% --------------------------------------------------------------------------
% Producto vectorial
% --------------------------------------------------------------------------
\item
  Use el producto vectorial para encontrar los componentes del vector unitario
  $\xhat{n}$ perpendicular al plano sombreado de la figura~\ref{fig:edcmat-prod-vect-n}.
  \vspace{-2ex}
  \begin{figure}[ht]
    \def\scl{.75}
    % Puntos de corte del plano con los ejes en la proporción [0-1]
    \pgfmathsetmacro{\CZ}{.7}
    \pgfmathsetmacro{\CY}{2*\CZ/3}
    \pgfmathsetmacro{\CX}{.4*\CZ}
    \centering
    \begin{tikzpicture}[%
      scale=\scl,
      eje/.style={->,line width=.8pt},
      vectorNormal/.style={-{Latex[round]}, line width=1.1pt},
      ]
      %
      % COORDENADAS
      % Origen
      \coordinate (O) at (0,0);
      % Extremos de los ejes
      \coordinate (ejeX) at (-1.0,-1.0);
      \coordinate (ejeY) at (2,0);
      \coordinate (ejeZ) at (0,2);
      % Puntos de corte del plano con los ejes
      \coordinate (corteX) at ($(O) !\CX! (ejeX)$);
      \coordinate (corteY) at ($(O) !\CY! (ejeY)$);
      \coordinate (corteZ) at ($(O) !\CZ! (ejeZ)$);
      % Origen del vector normal
      \coordinate (on) at (.3,.3);
      %
      % DIBUJO
      % Ejes
      \draw[eje] (O) -- (ejeX) node[left]{$x$};
      \draw[eje] (O) -- (ejeY) node[below=3pt]{$y$};
      \draw[eje] (O) -- (ejeZ) node[left=3pt] {$z$};
      % Plano
      %\filldraw[fill=green!20, draw=black, opacity=.8] (corteX) -- (corteY) -- (corteZ) -- cycle;
      \filldraw[fill=black!15, draw=black, opacity=.8] (corteX) -- (corteY) -- (corteZ) -- cycle;
      \node[below] at (corteX) {\small $1$};
      \node[below] at (corteY) {\small $2$};
      \node[left] at (corteZ) {\small $3$};
      % Vector normal al plano
      \draw[vectorNormal] (on) -- (40:1.5) node[right] {$\xhat{n}$};
      \filldraw (on) circle[radius=1pt];
    \end{tikzpicture}
    \caption{}
    \label{fig:edcmat-prod-vect-n}
  \end{figure}

  \vspace{-1ex}
  % ...........................................................................
  {\footnotesize \textcolor{gray}{[No resuelto]}}
  %\showSolved{res/matematicas}{matematicas-004.pdf}
% ...........................................................................
\medskip
{\color{gray}
\hrule
}

% --------------------------------------------------------------------------
%  Producto vectorial
% --------------------------------------------------------------------------
\item Demuestre
  $\vvv{A} \times (\vvv{B}\times\vvv{C})
  = \vvv{B}(\vvv{A}\cdot\vvv{C}) - \vvv{C}(\vvv{A}\cdot\vvv{B})$ escribiendo
  ambos miembros en términos de las componentes de los vectores.

  % ...........................................................................
  {\footnotesize \textcolor{gray}{[No resuelto]}}
  %\showSolved{res/matematicas}{matematicas-005.pdf}
% ...........................................................................
\medskip
{\color{gray}
\hrule
}

% --------------------------------------------------------------------------
%  Producto vectorial
% --------------------------------------------------------------------------
\item Demuestre que
  $[\vvv{A}\times (\vvv{B}\times\vvv{C})]
  + [\vvv{B}\times (\vvv{C}\times\vvv{A})]
  + [\vvv{C}\times (\vvv{A}\times\vvv{B})]
  = 0$
  
  ¿En qué condiciones se cumple
  $\vvv{A}\times (\vvv{B}\times\vvv{C}) = (\vvv{A}\times\vvv{B})\times\vvv{C}$?

  % ...........................................................................
  {\footnotesize \textcolor{gray}{[No resuelto]}}
  %\showSolved{res/matematicas}{matematicas-006.pdf}
% ...........................................................................
\medskip
{\color{gray}
\hrule
}

% --------------------------------------------------------------------------
%  Vector de posición
% --------------------------------------------------------------------------
\item Encuentre el vector separación $\brcurs$ entre el punto fuente $(2,8,7)$ al
  punto del campo $(4,6,8)$. Determine su módulo $\rcurs$ y construya el vector
  unitario $\hrcurs$ correspondiente.

  % ...........................................................................
  {\footnotesize \textcolor{gray}{[No resuelto]}}
  %\showSolved{res/matematicas}{matematicas-007.pdf}
% ...........................................................................
\medskip
{\color{gray}
\hrule
}

% --------------------------------------------------------------------------
%  Matrices de rotación
% --------------------------------------------------------------------------
\item
  \begin{subejercicio}
  \item Pruebe que la matriz de rotación $2\times 2$
    \[
      \begin{pNiceMatrix}
        \overline{A}_y\\
        \overline{A}_z
      \end{pNiceMatrix}
      =
      \begin{pNiceMatrix}
        \cos\phi & \sin\phi\\
        -\sin\phi & \cos\phi
      \end{pNiceMatrix}
      \begin{pNiceMatrix}
        A_y\\
        A_z
      \end{pNiceMatrix}
    \]
    preserva el producto escalar. Esto es, muestre que
    \[
    \overline{A}_y\,\, \overline{B}_y + \overline{A}_z\,\, \overline{B}_z
    = A_y B_y + A_z B_z
  \]
\item ¿Qué restricciones deben cumplir los elementos $R_{ij}$ de la matriz
  de rotación en tres dimensiones
  \[
    \begin{pNiceMatrix}
      \overline{A}_x\\
      \overline{A}_y\\
      \overline{A}_z
    \end{pNiceMatrix}
    =
    \begin{pNiceMatrix}
      R_{xx} & R_{xy} & R_{xz}\\
      R_{yx} & R_{yy} & R_{yz}\\
      R_{zx} & R_{zy} & R_{zz}\\
    \end{pNiceMatrix}
    \begin{pNiceMatrix}
      A_x\\
      A_y\\
      A_z
    \end{pNiceMatrix}
  \]
  para que se conserve la longitud de $\vvv{A}$ (para todos los vectores $\vvv{A}$)?
  \end{subejercicio}

  % ...........................................................................
  {\footnotesize \textcolor{gray}{[No resuelto]}}
  %\showSolved{res/matematicas}{matematicas-008.pdf}
% ...........................................................................
%\medskip
%{\color{gray}
%\hrule
%}

% #########################################################################################
% #########################################################################################
% #########################################################################################

\clearpage
% \phantomsection
%\addcontentsline{toc}{subsection}{Hoja 1}
%\setcounter{isubsheet}{2}
\stepcounter{isubsheet} 

% Añade 'Contenidos' al índice pdf.
% \bookmark[level=2,dest=section]{Hoja 1}
% Nombre de enlace: 'cc1.2'
\pdfbookmark[2]{Hoja 2}{mathj02}

% --------------------------------------------------------------------------
%  Matriz de rotación
% --------------------------------------------------------------------------
\item Encuentre la matriz de transformación $\mmm{R}$ que describa una rotación
  de \ang{120} alrededor de un eje que pase por el origen y por el punto $(1,1,1)$.
  La rotación debe ser antihoraria en cuando se observa el eje hacia el origen.

  % ...........................................................................
  {\footnotesize \textcolor{gray}{[No resuelto]}}
  %\showSolved{res/matematicas}{matematicas-009.pdf}
% ...........................................................................
\medskip
{\color{gray}
\hrule
}

% --------------------------------------------------------------------------
%  
% --------------------------------------------------------------------------
\item
  \begin{subejercicio}
  \item ¿Cómo se transforman las componentes de un vector\footnotemark{} bajo una \emph{transformación} de
    coordenadas ($\overline{x} = x$, $\overline{y} = -a$, $\overline{z} = z$,
    Fig.~\ref{fig:edcmat-dos-SR-separados})?
    \footnotetext{Un escalar no varía al cambiar las coordenadas. En particular, las componentes de un
      vector no son escalares, pero su módulo sí lo es.\\}
  \item ¿Cómo se transforman las componentes de un vector bajo una \emph{inversión} de
    coordenada ($\overline{x} = x$, $\overline{y} = -a$, $\overline{z} = z$,
    Fig.~\ref{fig:edcmat-dos-SR-origen-comun})?
  \item ¿Cómo se transforman las componentes de un producto vectorial
    \[
      \vvv{A}\times\vvv{B}
      = (A_y B_z - A_z B_y)\xhat{x} + (A_z B_x - A_x B_z)\xhat{y} + (A_x B_y - A_y B_x)\xhat{z}
    \]
    bajo una inversión de coordenadas
    (el producto vectorial de dos vectores se denomina \emph{pseudovector} a causa de este
    comportamiento \emph{anómalo})?
    Es el producto vectorial de dos pseudovectores un vector o pseudovector? Nombra dos
    magnitudes que sean pseudovectores en mecánica clásica.
  \item ¿Cómo se transforma el producto triple de vectores bajo inversiones (este objeto se
    llama \emph{pseudoescalar})?
  \end{subejercicio}
  \begin{figure}[ht]
      \def\scl{1}
      \pgfmathsetmacro{\XLEN}{1.5}
      \pgfmathsetmacro{\YLEN}{2.5}
      \pgfmathsetmacro{\ZLEN}{1.5}
      \pgfmathsetmacro{\YDESPLONE}{.9}
      \pgfmathsetmacro{\ORIGRADIUS}{.6}
      \centering
      \begin{minipage}{.45\linewidth}
        \centering
      \begin{tikzpicture}[%
        scale=\scl,
        eje/.style={->,thick},
        ]
        %
        % COORDENADAS
        % SISTEMA DE COORDENADAS ORIGINAL O
        \coordinate (O) at (0,0);
        \path (O) -- +(225:\XLEN) coordinate (ejex);
        \coordinate (ejey) at ($(O) + (\YLEN,0)$);
        \coordinate (ejez) at ($(O) + (0,\ZLEN)$);
        % SISTEMA DE COORDENADAS DESPLAZADO O'
        \coordinate (O') at (\YDESPLONE, 0);
        \coordinate (ejex') at ($(ejex) + (\YDESPLONE,0)$);
        \coordinate (ejey') at ($(ejey) + (\YDESPLONE,0)$);
        \coordinate (ejez') at ($(ejez) + (\YDESPLONE,0)$);
        %
        % DIBUJO
        % SC ORIGINAL
        \draw[eje] (O) -- (ejex) node[above left=-2pt and 0pt] {$x$};
        \draw[eje] (O) -- (ejey) node[below left=2pt and 0pt] {$y$};
        \draw[eje] (O) -- (ejez) node[below left=0pt and 2pt] {$z$};
        \filldraw (O) circle[radius=\ORIGRADIUS pt];
        % SC DESPLAZADO
        \draw[eje] (O') -- (ejex') node[above left=-2pt and 0pt] {$\overline{x}$};
        \draw[eje] (O') -- (ejey') node[below left=0pt and 0pt] {$\overline{y}$};
        \draw[eje] (O') -- (ejez') node[below left=-3pt and 2pt] {$\overline{z}$};
        \filldraw (O') circle[radius=\ORIGRADIUS pt];
        % Separation distance
        \path (O) -- node[midway,above] {$a$} (O');
      \end{tikzpicture}
      \caption{}
      \label{fig:edcmat-dos-SR-separados}
    \end{minipage}
    \hspace{1em}
    \begin{minipage}{.45\linewidth}
      \pgfmathsetmacro{\XLEN}{1.5}
      \pgfmathsetmacro{\YLEN}{2.0}
      \pgfmathsetmacro{\ZLEN}{1.5}
      \pgfmathsetmacro{\YDESPLTWO}{-.15}
      \pgfmathsetmacro{\ORIGRADIUS}{.6}
      \centering
      \begin{tikzpicture}[%
        scale=\scl,
        eje/.style={->,thick},
        ]
        % 
        % COORDENADAS
        % SISTEMA DE COORDENADAS ORIGINAL O
        \coordinate (O) at (0,0);
        \path (O) -- +(225:\XLEN) coordinate (ejex);
        \coordinate (ejey) at ($(O) + (\YLEN,0)$);
        \coordinate (ejez) at ($(O) + (0,\ZLEN)$);
        % SISTEMA DE COORDENADAS DESPLAZADO O'
        \coordinate (O') at (\YDESPLTWO, 0);
        \path (O') -- +(45:\XLEN) coordinate (ejex');
        \coordinate (ejey') at ($(O') + (-\YLEN,0)$);
        \coordinate (ejez') at ($(O') + (0,-\ZLEN)$);
        % 
        % DIBUJO
        % SC ORIGINAL
        \draw[eje] (O) -- (ejex) node[above left=-2pt and 0pt] {$x$};
        \draw[eje] (O) -- (ejey) node[below left=2pt and 0pt] {$y$};
        \draw[eje] (O) -- (ejez) node[below left=0pt and 2pt] {$z$};
        \filldraw (O) circle[radius=\ORIGRADIUS pt];
        % SC DESPLAZADO
        \draw[eje] (O') -- (ejex') node[below right=-2pt and 0pt] {$\overline{x}$};
        \draw[eje] (O') -- (ejey') node[below right=0pt and 0pt] {$\overline{y}$};
        \draw[eje] (O') -- (ejez') node[above right=0pt and 2pt] {$\overline{z}$};
        \filldraw (O') circle[radius=\ORIGRADIUS pt];
      \end{tikzpicture}
      \caption{}
      \label{fig:edcmat-dos-SR-origen-comun}
    \end{minipage}
  \end{figure}
  
  % ...........................................................................
  {\footnotesize \textcolor{gray}{[No resuelto]}}
  % \showSolved{res/matematicas}{matematicas-010.pdf}
  % ...........................................................................
  \medskip
  {\color{gray}
    \hrule
  }
  
  % --------------------------------------------------------------------------
  % Gradiente
  % --------------------------------------------------------------------------
\item Encuentre el gradiente de las siguientes funciones:
  \begin{subejercicio}
  \item $f(x,y,z) = x^2 + y^3 + z^4$.
  \item $f(x,y,z) = x^2 y^3 z^4$.
  \item $f(x,y,z) = e^x \sin y \ln z$.
  \end{subejercicio}
  % ...........................................................................
  {\footnotesize \textcolor{gray}{[No resuelto]}}
  % \showSolved{res/matematicas}{matematicas-011.pdf}
  % ...........................................................................
  \medskip
  {\color{gray}
    \hrule
  }
  
  % --------------------------------------------------------------------------
  % Gradiente
  % --------------------------------------------------------------------------
\item La altura de una colina (en pies) está dada por
  \[
    h(x,y) = 10 (2xy - 3x^2 - 4y^2 - 18x + 28y + 12
  \]
  donde $x$ e $y$ son las distancias (en millas) desde South Hadley.
  
  \begin{subejercicio}
  \item ¿Dónde está situado el pico de la colina?
  \item ¿Qué altura tiene este pico?
  \item ¿Cuál es la pendiente (en pies por milla) en un punto situado una
    milla al norte y una milla al este de South Hadley?
    ¿En qué dirección es más pronunciada la pendiente en ese punto?
  \end{subejercicio}
  % ...........................................................................
  {\footnotesize \textcolor{gray}{[No resuelto]}}
  % \showSolved{res/matematicas}{matematicas-012.pdf}
  % ...........................................................................
  \medskip
  {\color{gray}
    \hrule
  }
  
  % --------------------------------------------------------------------------
  % Gradiente
  % --------------------------------------------------------------------------
\item Sea $\brcurs$ el vector que separa un punto fijo $(x',y',z')$ del
  punto $(x,y,z)$, y sea $\rcurs$ su longitud. Compruebe que
  \begin{subejercicio}
  \item $\vvv{\nabla} \rcurs^2 = 2\brcurs$.
  \item $\vvv{\nabla} 1/\rcurs = -\hrcurs/\rcurs^2$.
  \item ¿Cuál es la fórmula general para $\vvv{\nabla} \rcurs^n$?
  \end{subejercicio}
  % ...........................................................................
  {\footnotesize \textcolor{gray}{[No resuelto]}}
  % \showSolved{res/matematicas}{matematicas-013.pdf}
  % ...........................................................................
  \medskip
  {\color{gray}
    \hrule
  }
  
  % --------------------------------------------------------------------------
  % Gradiente
  % --------------------------------------------------------------------------
\item Suponga que $f$ es una función de dos variables, $y$ y $z$.
  Demuestre que el gradiente
  $\vvv{\nabla} f = (\partial f/\partial y)\xhat{y} + (\partial f/\partial z)\xhat{z}$
  se transforma como un vector bajo rotaciones.
  \[
    \begin{pNiceMatrix}
      \overline{A}_y\\
      \overline{A}_z
    \end{pNiceMatrix}
    =
    \begin{pNiceMatrix}
      \cos\phi & \sin\phi\\
      -\sin\phi & \cos\phi
    \end{pNiceMatrix}
    \begin{pNiceMatrix}
      A_y\\
      A_z
    \end{pNiceMatrix}
  \]
  [Pista:
  $(\partial f/\partial\overline{y})(\partial y/\partial\overline{y})
  + (\partial f/\partial\overline{z}) (\partial z/\partial\overline{z})$,
  y la fórmula análoga para $\partial f/\partial\overline{z}$. Sabemos que
  $\overline{y} = y\cos\phi + z\sin\phi$ y $\overline{z} = -y\sin\phi + z\cos\phi$;
  ``resuelva'' estas ecuaciones para $y$ y $z$ (como funciones de $\overline{y}$ y
  $\overline{z}$), y calcule las derivadas intermedias $\partial y/\partial\overline{y}$,
  $\partial z/\partial\overline{y}$, etc.]
  
  % ...........................................................................
  {\footnotesize \textcolor{gray}{[No resuelto]}}
  % \showSolved{res/matematicas}{matematicas-014.pdf}
  % ...........................................................................
  % \medskip
  % {\color{gray}
  % \hrule
  % }
  
  %   #########################################################################################
  %   #########################################################################################
  %   #########################################################################################
  
  \clearpage
  % \phantomsection
  % \addcontentsline{toc}{subsection}{Hoja 1}
  %\setcounter{isubsheet}{3}
  \stepcounter{isubsheet} 
  
  % Añade 'Contenidos' al índice pdf.
  % \bookmark[level=2,dest=section]{Hoja 1}
  % Nombre de enlace: 'cc1.2'
  \pdfbookmark[2]{Hoja 3}{mathj03}
  
  % --------------------------------------------------------------------------
  % Divergencia
  % --------------------------------------------------------------------------
\item Calcule la divergencia de las siguientes funciones vectoriales:
  \begin{subejercicio}
  \item $\vvv{v}_a = x^2 \xhat{x} + 3xz^2 \xhat{y} - 2xz \xhat{z}$.
  \item $\vvv{v}_b = xy \xhat{x} + 2yz \xhat{y} + 3zx \xhat{z}$.
  \item $\vvv{v}_c = y^2 \xhat{x} + (2xy+z^2) \xhat{y} + 2yz \xhat{z}$.
  \end{subejercicio}
  
  % ...........................................................................
  {\footnotesize \textcolor{gray}{[No resuelto]}}
  % \showSolved{res/matematicas}{matematicas-015.pdf}
  % ...........................................................................
  \medskip
  {\color{gray}
    \hrule
  }
  
  % --------------------------------------------------------------------------
  % Divergencia
  % --------------------------------------------------------------------------
\item Represente un boceto de la función vectorial
  $\vvv{v} = \xhat{r}/r^2$
  % \[
  %   \vvv{v} = \frac{\xhat{r}}{r^2}
  % \]
  y calcule su divergencia. La solución le podría desconcertar, ¿puede dar una explicación?
  
  % ...........................................................................
  {\footnotesize \textcolor{gray}{[No resuelto]}}
  % \showSolved{res/matematicas}{matematicas-016.pdf}
  % ...........................................................................
  \medskip
  {\color{gray}
    \hrule
  }
  
  % --------------------------------------------------------------------------
  % Divergencia
  % --------------------------------------------------------------------------
\item Demuestre que en dos dimensiones, la divergencia se transforma como un escalar bajo
  rotaciones.\\
  (Pista: Use la ecuación para calcular $\bar{v}_y$ y $\bar{v}_z$ y calcule sus derivadas
  parciales.\\
  Se debe demostrar que
  $\partial\bar{v}_y/\partial\bar{y} + \partial\bar{v}_z/\partial\bar{z}
  = \partial v_y/\partial y + \partial v_z/\partial z$
  
  % ...........................................................................
  {\footnotesize \textcolor{gray}{[No resuelto]}}
  % \showSolved{res/matematicas}{matematicas-017.pdf}
  % ...........................................................................
  \medskip
  {\color{gray}
    \hrule
  }
  
  % --------------------------------------------------------------------------
  % Rotacional
  % --------------------------------------------------------------------------
\item Calcule el rotacional de las siguientes funciones vectoriales:
  \begin{subejercicio}
  \item $\vvv{v}_a = x^2 \xhat{x} + 3xz^2 \xhat{y} - 2xz \xhat{z}$.
  \item $\vvv{v}_b = xy \xhat{x} + 2yz \xhat{y} + 3zx \xhat{z}$.
  \item $\vvv{v}_c = y^2 \xhat{x} + (2xy+z^2) \xhat{y} + 2yz \xhat{z}$.
  \end{subejercicio}
  
  % ...........................................................................
  {\footnotesize \textcolor{gray}{[No resuelto]}}
  % \showSolved{res/matematicas}{matematicas-018.pdf}
  % ...........................................................................
  \medskip
  {\color{gray}
    \hrule
  }
  
  % --------------------------------------------------------------------------
  % Rotacional
  % --------------------------------------------------------------------------
\item Dibuje un círculo en el plano $xy$. En unos cuantos puntos representativos dibuje
  el vector $\vvv{v}$ tangente al círculo en sentido antihorario. Comparando vectores
  adyacentes, determine el signo de $\partial v_x/\partial y$ y $\partial v_y/\partial x$.
  De acuerdo con la ecuación
  \vspace{-.5ex}
  \[
    \vvv{\nabla}\times\vvv{v}
    =
    \begin{vNiceMatrix}
      \xhat{x} & \xhat{y} & \xhat{z}\\
      \partial/\partial x & \partial/\partial y & \partial/\partial z\\
      v_x & v_y & v_z \\
    \end{vNiceMatrix}
  \]
  ¿Cuál será la dirección y sentido de $\vvv{\nabla}\times\vvv{v}$?
  Explique cómo este ejemplo ilustra la interpretación geométrica del rotacional.
  
  % ...........................................................................
  {\footnotesize \textcolor{gray}{[No resuelto]}}
  % \showSolved{res/matematicas}{matematicas-019.pdf}
  % ...........................................................................
  \medskip
  {\color{gray}
    \hrule
  }
  
  % --------------------------------------------------------------------------
  % Divergencia y Rotacional
  % --------------------------------------------------------------------------
\item Construya una función vectorial cuya divergencia y rotacional sean nulos
  en todo punto. Una función constante serviría, pero esperamos que presente
  una función un poco más interesante.
  
  % ...........................................................................
  {\footnotesize \textcolor{gray}{[No resuelto]}}
  % \showSolved{res/matematicas}{matematicas-020.pdf}
  % ...........................................................................
  \medskip
  {\color{gray}
    \hrule
  }
  
  % --------------------------------------------------------------------------
  % Gradiente, divergencia y rotacional
  % --------------------------------------------------------------------------
\item Demuestre las siguientes propiedades:
  \begin{subejercicio}
  \item $\vvv{\nabla} (fg) = f\vvv{\nabla} g + g\vvv{\nabla} f$
  \item $\vvv{\nabla}\cdot (\vvv{A}\times\vvv{B})
    = \vvv{B}\cdot(\vvv{\nabla}\times\vvv{A}) - \vvv{A}\cdot(\vvv{\nabla}\times\vvv{B})$
  \item $\vvv{\nabla}\times (f\vvv{A})
    = f(\vvv{\nabla}\times\vvv{A}) - \vvv{A}\times (\vvv{\nabla} f)$
  \end{subejercicio}
  
  % ...........................................................................
  {\footnotesize \textcolor{gray}{[No resuelto]}}
  % \showSolved{res/matematicas}{matematicas-021.pdf}
  % ...........................................................................
  \medskip
  {\color{gray}
    \hrule
  }
  
  % --------------------------------------------------------------------------
  % 
  % --------------------------------------------------------------------------
\item 
  \begin{subejercicio}
  \item Si $\vvv{A}$ y $\vvv{B}$ son dos funciones vectoriales, ¿qué significa
    la expresión $(\vvv{A}\cdot\vvv{\nabla})\vvv{B}$?
    (Esto es, ¿cuáles son sus componentes $x$, $y$ y $z$ en términos de las
    componentes cartesianas de $\vvv{A}$, $\vvv{B}$ y $\vvv{\nabla}$?)
  \item Calcule $(\xhat{r}\cdot\vvv{\nabla})\xhat{r}$.
  \item Calcule $(\vvv{v}_a\cdot\vvv{\nabla})\vvv{v}_b$ para las funciones vectoriales
    $\vvv{v}_a = x^2 \xhat{x} + 3xz^2 \xhat{y} - 2xz \xhat{z}$,
    $\vvv{v}_b = xy \xhat{x} + 2yz \xhat{y} + 3zx \xhat{z}$ y
    $\vvv{v}_c = y^2 \xhat{x} + (2xy+z^2) \xhat{y} + 2yz \xhat{z}$.
  \end{subejercicio}
  
  % ...........................................................................
  {\footnotesize \textcolor{gray}{[No resuelto]}}
  % \showSolved{res/matematicas}{matematicas-022.pdf}
  % ...........................................................................
  \medskip
  {\color{gray}
    \hrule
  }
  
  % --------------------------------------------------------------------------
  % 
  % --------------------------------------------------------------------------
\item (Sólo para masoquistas). Demuestre que se cumplen las igualdades:
  \begin{subejercicio}
  \item
    $\vvv{\nabla}(\vvv{A}\cdot\vvv{B})
    = \vvv{A}\times (\vvv{\nabla}\times\vvv{B}
    + \vvv{B}\times (\vvv{\nabla}\times\vvv{A}
    + (\vvv{A}\cdot\vvv{\nabla})\vvv{B}
    + (\vvv{B}\cdot\vvv{\nabla})\vvv{A}$
  \item
    $\vvv{\nabla}\times (\vvv{A}\times\vvv{B})
    = (\vvv{B}\cdot\vvv{\nabla})\vvv{A}
    - (\vvv{A}\cdot\vvv{\nabla})\vvv{B}
    + \vvv{A} (\vvv{\nabla}\cdot\vvv{B})
    - \vvv{B} (\vvv{\nabla}\cdot\vvv{A})$
  \end{subejercicio}
  
  % ...........................................................................
  {\footnotesize \textcolor{gray}{[No resuelto]}}
  % \showSolved{res/matematicas}{matematicas-023.pdf}
  % ...........................................................................
  \medskip
  {\color{gray}
    \hrule
  }
  
  % --------------------------------------------------------------------------
  % 
  % --------------------------------------------------------------------------
\item Demuestre las siguientes igualdades
  \smallskip
  \begin{subejercicio}
  \item
    $\vvv{\nabla}\left(\dfrac{f}{g}\right)
    = \dfrac{g\vvv{\nabla}f - f\vvv{\nabla}g}{g^2}$\\[.3ex]
  \item
    $\vvv{\nabla}\cdot\left(\dfrac{\vvv{\nabla}}{g}\right)
    = \dfrac{g(\vvv{\nabla}\cdot\vvv{A}) - \vvv{A}\cdot(\vvv{\nabla}g)}{g^2}$\\[.3ex]
  \item
    $\vvv{\nabla}\times\left(\dfrac{\vvv{\nabla}}{g}\right)
    = \dfrac{g(\vvv{\nabla}\times\vvv{A}) + \vvv{A}\times(\vvv{\nabla}g)}{g^2}$
  \end{subejercicio}

  \medskip
  % ...........................................................................
  {\footnotesize \textcolor{gray}{[No resuelto]}}
  % \showSolved{res/matematicas}{matematicas-024.pdf}
  % ...........................................................................
  \medskip
  % \medskip
  % {\color{gray}
  % \hrule
  % }
  
  %   #########################################################################################
  %   #########################################################################################
  %   #########################################################################################
  
  \clearpage
  % \phantomsection
  % \addcontentsline{toc}{subsection}{Hoja 1}
  %\setcounter{isubsheet}{3}
  \stepcounter{isubsheet} 
  
  % Añade 'Contenidos' al índice pdf.
  % \bookmark[level=2,dest=section]{Hoja 1}
  % Nombre de enlace: 'cc1.2'
  \pdfbookmark[2]{Hoja 4}{mathj04}
  
  % --------------------------------------------------------------------------
  % 
  % --------------------------------------------------------------------------
\item 
  \begin{subejercicio}
  \item Compruebe la propiedad
    $\vvv{\nabla}\cdot (\vvv{A}\times\vvv{B})
    = \vvv{B}\cdot (\vvv{\nabla}\times\vvv{A}) - \vvv{A}\cdot (\vvv{\nabla}\times\vvv{B})$
    calculando cada término (por separado) para las funciones
    $\vvv{A} = x\xhat{x} + 2y\xhat{y} + 3z\xhat{z}$ y
    $\vvv{B} = 3y\xhat{x} - 2x\xhat{y}$.
  \item Haga lo mismo para la propiedad
    $\vvv{\nabla}(\vvv{A}\cdot\vvv{B})
    = \vvv{A}\times (\vvv{\nabla}\times\vvv{B}
    + \vvv{B}\times (\vvv{\nabla}\times\vvv{A}
    + (\vvv{A}\cdot\vvv{\nabla})\vvv{B}
    + (\vvv{B}\cdot\vvv{\nabla})\vvv{A}$
  \item Vuelva a hacer lo mismo para la propiedad
    $\vvv{\nabla}\times (\vvv{A}\times\vvv{B})
    = (\vvv{B}\cdot\vvv{\nabla})\vvv{A}
    - (\vvv{A}\cdot\vvv{\nabla})\vvv{B}
    + \vvv{A} (\vvv{\nabla}\cdot\vvv{B})
    - \vvv{B} (\vvv{\nabla}\cdot\vvv{A})$    
  \end{subejercicio}
  
  % ...........................................................................
  {\footnotesize \textcolor{gray}{[No resuelto]}}
  % \showSolved{res/matematicas}{matematicas-025.pdf}
  % ...........................................................................
  \medskip
  {\color{gray}
    \hrule
  }
  
  % --------------------------------------------------------------------------
  % Laplaciano
  % --------------------------------------------------------------------------
\item Calcule el laplaciano de las siguientes funciones
  \begin{subejercicio}
  \item $T_a = x^2 + 2xy + 3z + 4$.
  \item $T_b = \sin x \sin y \sin z$.
  \item $T_c = e^{-5x} \sin 4y \cos 3z$.
  \item $\vvv{v} = x^2\xhat{x} + 3xz^2\xhat{y} - 2xz\xhat{z}$.
  \end{subejercicio}
  
  % ...........................................................................
  {\footnotesize \textcolor{gray}{[No resuelto]}}
  % \showSolved{res/matematicas}{matematicas-026.pdf}
  % ...........................................................................
  \medskip
  {\color{gray}
    \hrule
  }
  
  % --------------------------------------------------------------------------
  % Divergencia de un rotacional
  % --------------------------------------------------------------------------
\item Pruebe que la divergencia de un rotacional siempre vale cero. Compruébelo
  para $\vvv{v}_a = x^2\xhat{x} + 3xz^2\xhat{y} - 2xz\xhat{z}$.
  
  % ...........................................................................
  {\footnotesize \textcolor{gray}{[No resuelto]}}
  % \showSolved{res/matematicas}{matematicas-027.pdf}
  % ...........................................................................
  \medskip
  {\color{gray}
    \hrule
  }
  
  % --------------------------------------------------------------------------
  % Rotacional de un gradiente
  % --------------------------------------------------------------------------
\item Pruebe que el rotacional de un gradiente siempre vale cero. Compruébelo
  para $f(x,y,z) = x^2 y^3 z^4$.
  
  % ...........................................................................
  {\footnotesize \textcolor{gray}{[No resuelto]}}
  % \showSolved{res/matematicas}{matematicas-028.pdf}
  % ...........................................................................
  \medskip
  {\color{gray}
    \hrule
  }
  
  % --------------------------------------------------------------------------
  % Integral de línea
  % --------------------------------------------------------------------------
\item Calcule la integral de línea de la función
  $\vvv{v} = x^2\xhat{x} + 2yz\xhat{y} + y^2\xhat{z}$ desde el origen hasta
  el punto $(1,1,1)$ a trevés de tres caminos diferentes:
  \begin{subejercicio}
  \item $(0,0,0) \longrightarrow (1,0,0)
    \longrightarrow (1,1,0) \longrightarrow (1,1,1).$
  \item $(0,0,0) \longrightarrow (0,0,1)
    \longrightarrow (0,1,1) \longrightarrow (1,1,1).$
  \item Siguiendo una línea recta desde $(0,0,0)$ hasta $(1,1,1)$.
  \item ¿Cuánto vale la integral de línea cuando se sigue el bucle cerrado
    cuando se sale del camino del apartado $a)$ y se vuelve por el camino $b)$?
  \end{subejercicio}
  
  % ...........................................................................
  {\footnotesize \textcolor{gray}{[No resuelto]}}
  % \showSolved{res/matematicas}{matematicas-029.pdf}
  % ...........................................................................
  \medskip
  {\color{gray}
    \hrule
  }
  
  % --------------------------------------------------------------------------
  % Integral de superfice
  % --------------------------------------------------------------------------
\item Calcule la integral de línea de la función
  $\vvv{v} = 2xz\xhat{x} + (x+2)\xhat{y} + y(z^2-3)\xhat{z}$ a través de todas
  las caras del cubo de \SI{2}{\metre} de arista de la figura~\ref{fig:edcmat-cubo1}
  (exceptuando la de la base).
  El sentido de cada una de las caras es hacia fuera del cubo.
  \vspace{-1ex}
  \begin{figure}[ht]
    \def\scl{.65}
    %\tdplotsetmaincoords{0}{0}
    \pgfmathsetmacro{\LONGEJEX}{2.6}
    \pgfmathsetmacro{\LONGEJEY}{2.2}
    \pgfmathsetmacro{\LONGEJEZ}{2.2}
    \pgfmathsetmacro{\LONGCUBO}{1.5}
    \centering
    \begin{tikzpicture}[%
      scale=\scl,
      %tdplot_main_coords,
      eje/.style={->,line width=.6pt,black},
      %cubo/.style={white,double=black,double distance=.1mm,join=bevel,line width=1.2pt},
      cubo/.style={line width=.85pt,black},
      cara opaca/.style={line width=.85,fill=black!2,draw=black},
      cara translucida/.style={line width=.85,fill=black!2,draw=black,opacity=.75},
      ]
      % Coordenadas
      \coordinate (A1) at (0,0,0);
      \coordinate (A2) at (0,\LONGCUBO,0);
      \coordinate (A3) at (\LONGCUBO,\LONGCUBO,0);
      \coordinate (A4) at (\LONGCUBO,0,0);
      \coordinate (B1) at (0,0,\LONGCUBO);
      \coordinate (B2) at (0,\LONGCUBO,\LONGCUBO);
      \coordinate (B3) at (\LONGCUBO,\LONGCUBO,\LONGCUBO);
      \coordinate (B4) at (\LONGCUBO,0,\LONGCUBO);
      % Punto central del cubo
      \coordinate (O) at ($(A1)!.5!(B3)$);
      % Punto cara trasera
      \coordinate (tra) at ($(A2)!.5!(A4)$);
      % Punto cara izquierda
      \coordinate (izda) at ($(A2)!.5!(B1)$);
      % Punto cara superior
      \coordinate (sup) at ($(A2)!.5!(B3)$);
      % Punto cara derecha
      \coordinate (dcha) at ($(A3)!.5!(B4)$);
      % Punto cara frontal
      \coordinate (fro) at ($(B2)!.5!(B4)$);
      % Extremos de los ejes de coordenadas
      \coordinate (ejex) at (0,0,\LONGEJEX);
      \coordinate (ejey) at (\LONGEJEY,0,0);
      \coordinate (ejez) at (0,\LONGEJEZ,0);
      % Ejes
      \draw[eje] (A1) -- (ejex) node[above left] {$x$};
      \draw[eje] (A1) -- (ejey) node[below=2pt] {$y$};
      \draw[eje] (A1) -- (ejez) node[left=2pt] {$z$};
      % Vectores normales a las caras (menos a la de abajo)
      \def\radius{1.5pt}
      \def\linewidth{1.1pt}
      % Vector normal cara izquierda
      \draw[-{Latex},line width=\linewidth] (izda) -- +(left:3em)
      node[left] {\scriptsize\scshape (iv)};
      \filldraw[fill=black,draw=black] (izda) circle[radius=\radius];
      % Vector normal cara trasera
      \draw[-{Latex},black,line width=\linewidth] ($(fro)+(0,.1,.1)$) -- ++($(fro)-1.6*(ejex)$)
      node[right] {\scriptsize\scshape (ii)};
      \filldraw[fill=black,draw=black] ($(fro)+(0,.1,.1)$) circle[radius=\radius];
      %\draw[-{Latex},black,line width=\linewidth] (tra) -- ++($(fro)-.85*(ejex)$);
      %\filldraw[fill=black,draw=black] (tra) circle[radius=\radius];      
      % Cubo
      % Cara trasera
      \filldraw[cara opaca] (A1) -- (A2) -- (A3) -- (A4) -- cycle;
      % Cara inferior
      \filldraw[cara opaca] (A1) -- (A4) -- (B4) -- (B1) -- cycle;
      % Cara izquierda
      \filldraw[cara opaca] (A1) -- (B1) -- (B2) -- (A2) -- cycle;
      % Cara derecha
      \filldraw[cara translucida] (A4) -- (B4) -- (B3) -- (A3) -- cycle;
      % Cara superior
      \filldraw[cara translucida] (A2) -- (A3) -- (B3) -- (B2) -- cycle;
      % Cara frontal
      \filldraw[cara translucida] (B1) -- (B2) -- (B3) -- (B4) -- cycle;
      % Vector normal cara derecha
      \draw[-{Latex},line width=\linewidth] (dcha) -- +(right:3em)
      node[right] {\scriptsize\scshape (iii)};
      \filldraw[fill=black,draw=black] (dcha) circle[radius=\radius];
      % Vector normal cara frontal
      \draw[-{Latex},black,line width=\linewidth] ($(fro)+(0,.1,.1)$) -- ++($(fro)+.85*(ejex)$)
      node[above right=-4pt and 6pt] {\scriptsize\scshape (i)};
      \filldraw[fill=black,draw=black] ($(fro)+(0,.1,.1)$) circle[radius=\radius];
      %\draw[-{Latex},black,line width=\linewidth] (fro) -- ++($(fro)+.85*(ejex)$);
      %\filldraw[fill=black,draw=black] (fro) circle[radius=\radius];
      % Vector normal cara superior
      \draw[-{Latex},line width=\linewidth] (sup) -- +(up:3em)
      node[right] {\scriptsize\scshape (v)};
      \filldraw[fill=black,draw=black] (sup) circle[radius=\radius];

      
      % Vértices del cubo
      %\filldraw (A2) circle[radius=1pt] node[above left,black] {\scriptsize $A2$};
      %\filldraw (A3) circle[radius=1pt] node[above right,black] {\scriptsize $A3$};
      %\filldraw (A4) circle[radius=1pt] node[above right,black] {\scriptsize $A4$};
      %\filldraw (B1) circle[radius=1pt] node[left,black] {\scriptsize $B1$};
      %\filldraw (B2) circle[radius=1pt] node[left,black] {\scriptsize $B2$};
      %\filldraw (B3) circle[radius=1pt] node[right,black] {\scriptsize $B3$};
      %\filldraw (B4) circle[radius=1pt] node[right,black] {\scriptsize $B4$};
    \end{tikzpicture}
    \caption{}
    \label{fig:edcmat-cubo1}
  \end{figure}

  \vspace{-1ex}
  % ...........................................................................
  {\footnotesize \textcolor{gray}{[No resuelto]}}
  % \showSolved{res/matematicas}{matematicas-030.pdf}
  % ...........................................................................
  \medskip
  {\color{gray}
    \hrule
  }
  
  % --------------------------------------------------------------------------
  % Integral de volumen
  % --------------------------------------------------------------------------
\item Calcule la integral de volumen de la función $T=z^2$ sobre el tetraedro
  de vértices $(0,0,0)$, $(1,0,0)$, $(0,1,0)$ y $(0,0,1)$.
  
  % ...........................................................................
  {\footnotesize \textcolor{gray}{[No resuelto]}}
  % \showSolved{res/matematicas}{matematicas-031.pdf}
  % ...........................................................................
  \medskip
  {\color{gray}
    \hrule
  }
  
  % --------------------------------------------------------------------------
  % Gradiente
  % --------------------------------------------------------------------------
\item Compruebe el teorema fundamental para gradientes, usando $T=x^2+4xy+2yz^3$,
  los puntos $\vvv{a} = (0,0,0)$, $\vvv{b} = (1,1,1)$ y los tres caminos representados
  en la figura~\ref{fig:edcmat-trayectorias1}.
  \begin{subejercicio}
  \item $(0,0,0) \longrightarrow (1,0,0)
    \longrightarrow (1,1,0) \longrightarrow (1,1,1)$
  \item $(0,0,0) \longrightarrow (0,0,1)
    \longrightarrow (0,1,1) \longrightarrow (1,1,1).$
  \item La trayectoria parabólica $z=x^2$; $y=x$.
  \end{subejercicio}
  \begin{figure}[ht]
    \def\scl{.65}
    %\tdplotsetmaincoords{0}{0}
    \pgfmathsetmacro{\LONGEJEX}{2.6}
    \pgfmathsetmacro{\LONGEJEY}{2.2}
    \pgfmathsetmacro{\LONGEJEZ}{2.2}
    \pgfmathsetmacro{\LONGCUBO}{1.5}
    \centering
    \begin{subfigure}[b]{.2\textwidth}
    %\centering
    \begin{tikzpicture}[%
      scale=\scl,
      eje/.style={->,line width=.6pt,black},
      trayecto/.style={-{Latex},line width=.8pt,shorten >=1pt},
      punto/.style={fill=black,draw=black},
      cubo/.style={line width=.85pt,black},
      cara opaca/.style={line width=.85,fill=black!2,draw=black},
      cara translucida/.style={line width=.85,fill=black!2,draw=black,opacity=.75},
      ]
      % Coordenadas
      \coordinate (A1) at (0,0,0);
      \coordinate (A2) at (0,\LONGCUBO,0);
      \coordinate (A3) at (\LONGCUBO,\LONGCUBO,0);
      \coordinate (A4) at (\LONGCUBO,0,0);
      \coordinate (B1) at (0,0,\LONGCUBO);
      \coordinate (B2) at (0,\LONGCUBO,\LONGCUBO);
      \coordinate (B3) at (\LONGCUBO,\LONGCUBO,\LONGCUBO);
      \coordinate (B4) at (\LONGCUBO,0,\LONGCUBO);
      % Punto central del cubo
      \coordinate (O) at ($(A1)!.5!(B3)$);
      % Punto cara trasera
      \coordinate (tra) at ($(A2)!.5!(A4)$);
      % Punto cara izquierda
      \coordinate (izq) at ($(A2)!.5!(B1)$);
      % Punto cara superior
      \coordinate (sup) at ($(A2)!.5!(B3)$);
      % Punto cara derecha
      \coordinate (der) at ($(A3)!.5!(B4)$);
      % Punto cara frontal
      \coordinate (fro) at ($(B2)!.5!(B4)$);
      % Extremos de los ejes de coordenadas
      \coordinate (ejex) at (0,0,\LONGEJEX);
      \coordinate (ejey) at (\LONGEJEY,0,0);
      \coordinate (ejez) at (0,\LONGEJEZ,0);
      % Ejes
      \draw[eje] (A1) -- (ejex) node[above left] {$x$};
      \draw[eje] (A1) -- (ejey) node[below=2pt] {$y$};
      \draw[eje] (A1) -- (ejez) node[left=2pt] {$z$};
      % Trayectorias
      \draw[trayecto] (A1) -- (B1);
      \draw[trayecto] (B1) -- (B4);
      \draw[trayecto,shorten >=1.7pt] (B4) -- (B3);
      % Punto
      \filldraw (B3) circle [radius=2pt];
    \end{tikzpicture}
    \caption{}
  \end{subfigure}
  \hspace{1em}
  \begin{subfigure}[b]{.2\textwidth}
    %\centering
    \begin{tikzpicture}[%
      scale=\scl,
      eje/.style={->,line width=.6pt,black},
      trayecto/.style={-{Latex},line width=.8pt,shorten >=1pt},
      punto/.style={fill=black,draw=black},
      cubo/.style={line width=.85pt,black},
      cara opaca/.style={line width=.85,fill=black!2,draw=black},
      cara translucida/.style={line width=.85,fill=black!2,draw=black,opacity=.75},
      ]
      % Coordenadas
      \coordinate (A1) at (0,0,0);
      \coordinate (A2) at (0,\LONGCUBO,0);
      \coordinate (A3) at (\LONGCUBO,\LONGCUBO,0);
      \coordinate (A4) at (\LONGCUBO,0,0);
      \coordinate (B1) at (0,0,\LONGCUBO);
      \coordinate (B2) at (0,\LONGCUBO,\LONGCUBO);
      \coordinate (B3) at (\LONGCUBO,\LONGCUBO,\LONGCUBO);
      \coordinate (B4) at (\LONGCUBO,0,\LONGCUBO);
      % Punto central del cubo
      \coordinate (O) at ($(A1)!.5!(B3)$);
      % Punto cara trasera
      \coordinate (tra) at ($(A2)!.5!(A4)$);
      % Punto cara izquierda
      \coordinate (izq) at ($(A2)!.5!(B1)$);
      % Punto cara superior
      \coordinate (sup) at ($(A2)!.5!(B3)$);
      % Punto cara derecha
      \coordinate (der) at ($(A3)!.5!(B4)$);
      % Punto cara frontal
      \coordinate (fro) at ($(B2)!.5!(B4)$);
      % Extremos de los ejes de coordenadas
      \coordinate (ejex) at (0,0,\LONGEJEX);
      \coordinate (ejey) at (\LONGEJEY,0,0);
      \coordinate (ejez) at (0,\LONGEJEZ,0);
      % Ejes
      \draw[eje] (A1) -- (ejex) node[above left] {$x$};
      \draw[eje] (A1) -- (ejey) node[below=2pt] {$y$};
      \draw[eje] (A1) -- (ejez) node[left=2pt] {$z$};
      % Trayectorias
      \draw[trayecto] (A1) -- (A2);
      \draw[trayecto] (A2) -- (A3);
      \draw[trayecto] (A3) -- (B3);
      % Punto
      \filldraw (B3) circle [radius=2pt];
    \end{tikzpicture}
    \caption{}
  \end{subfigure}
  \hspace{1em}
  \begin{subfigure}[b]{.2\textwidth}
    \centering
    \begin{tikzpicture}[%
      scale=\scl,
      eje/.style={->,line width=.6pt,black},
      trayecto/.style={-{Latex},line width=.8pt,shorten >=1pt},
      punto/.style={fill=black,draw=black},
      cubo/.style={line width=.85pt,black},
      cara opaca/.style={line width=.85,fill=black!2,draw=black},
      cara translucida/.style={line width=.85,fill=black!2,draw=black,opacity=.75},
      ]
      % Coordenadas
      \coordinate (A1) at (0,0,0);
      \coordinate (A2) at (0,\LONGCUBO,0);
      \coordinate (A3) at (\LONGCUBO,\LONGCUBO,0);
      \coordinate (A4) at (\LONGCUBO,0,0);
      \coordinate (B1) at (0,0,\LONGCUBO);
      \coordinate (B2) at (0,\LONGCUBO,\LONGCUBO);
      \coordinate (B3) at (\LONGCUBO,\LONGCUBO,\LONGCUBO);
      \coordinate (B4) at (\LONGCUBO,0,\LONGCUBO);
      % Punto central del cubo
      \coordinate (O) at ($(A1)!.5!(B3)$);
      % Punto cara trasera
      \coordinate (tra) at ($(A2)!.5!(A4)$);
      % Punto cara izquierda
      \coordinate (izq) at ($(A2)!.5!(B1)$);
      % Punto cara superior
      \coordinate (sup) at ($(A2)!.5!(B3)$);
      % Punto cara derecha
      \coordinate (der) at ($(A3)!.5!(B4)$);
      % Punto cara frontal
      \coordinate (fro) at ($(B2)!.5!(B4)$);
      % Extremos de los ejes de coordenadas
      \coordinate (ejex) at (0,0,\LONGEJEX);
      \coordinate (ejey) at (\LONGEJEY,0,0);
      \coordinate (ejez) at (0,\LONGEJEZ,0);
      % Ejes
      \draw[eje] (A1) -- (ejex) node[above left] {$x$};
      \draw[eje] (A1) -- (ejey) node[below=2pt] {$y$};
      \draw[eje] (A1) -- (ejez) node[left=2pt] {$z$};
      % Trayectorias
      \draw[trayecto] (A1) parabola (B3);
      %\draw[trayecto] (A1) -- (A2);
      %\draw[trayecto] (A2) -- (A3);
      %\draw[trayecto] (A3) -- (B3);
      % Punto
      \filldraw (B3) circle [radius=2pt];
    \end{tikzpicture}
    \caption{}
  \end{subfigure}
  \caption{}
  \label{fig:edcmat-trayectorias1}
\end{figure}

\vspace{-.5ex}
  % ...........................................................................
  {\footnotesize \textcolor{gray}{[No resuelto]}}
  % \showSolved{res/matematicas}{matematicas-032.pdf}
  % ...........................................................................
  %\medskip
  %{\color{gray}
  %  \hrule
  %}
  
  %   #########################################################################################
  %   #########################################################################################
  %   #########################################################################################
  
  \clearpage
  % \phantomsection
  % \addcontentsline{toc}{subsection}{Hoja 1}
  %\setcounter{isubsheet}{3}
  \stepcounter{isubsheet} 
  
  % Añade 'Contenidos' al índice pdf.
  % \bookmark[level=2,dest=section]{Hoja 1}
  % Nombre de enlace: 'cc1.2'
  \pdfbookmark[2]{Hoja 5}{mathj05}
  
  % --------------------------------------------------------------------------
  % Teorema de la divergencia
  % --------------------------------------------------------------------------
\item Compruebe el teorema de la divergencia para la función
    $\vvv{v} = (xy) \xhat{x} + (2yz) \xhat{y} + (3zx) \xhat{z}$.
    Utilice el volumen del cubo de \SI{2}{\metre} de arista de la
    figura~\ref{fig:edcmat-cubo2}.
    \vspace{-2ex}
    \begin{figure}[ht]
    \def\scl{.65}
    %\tdplotsetmaincoords{0}{0}
    \pgfmathsetmacro{\LONGEJEX}{2.6}
    \pgfmathsetmacro{\LONGEJEY}{2.2}
    \pgfmathsetmacro{\LONGEJEZ}{2.2}
    \pgfmathsetmacro{\LONGCUBO}{1.5}
    \centering
    \begin{tikzpicture}[%
      scale=\scl,
      eje/.style={->,line width=.6pt,black},
      cubo/.style={line width=.85pt,black},
      cara opaca/.style={line width=.85,fill=black!2,draw=black},
      cara translucida/.style={line width=.85,fill=black!2,draw=black,opacity=.75},
      ]
      % Coordenadas
      \coordinate (A1) at (0,0,0);
      \coordinate (A2) at (0,\LONGCUBO,0);
      \coordinate (A3) at (\LONGCUBO,\LONGCUBO,0);
      \coordinate (A4) at (\LONGCUBO,0,0);
      \coordinate (B1) at (0,0,\LONGCUBO);
      \coordinate (B2) at (0,\LONGCUBO,\LONGCUBO);
      \coordinate (B3) at (\LONGCUBO,\LONGCUBO,\LONGCUBO);
      \coordinate (B4) at (\LONGCUBO,0,\LONGCUBO);
      % Punto central del cubo
      \coordinate (O) at ($(A1)!.5!(B3)$);
      % Punto cara trasera
      \coordinate (tra) at ($(A2)!.5!(A4)$);
      % Punto cara izquierda
      \coordinate (izq) at ($(A2)!.5!(B1)$);
      % Punto cara superior
      \coordinate (sup) at ($(A2)!.5!(B3)$);
      % Punto cara derecha
      \coordinate (der) at ($(A3)!.5!(B4)$);
      % Punto cara frontal
      \coordinate (fro) at ($(B2)!.5!(B4)$);
      % Extremos de los ejes de coordenadas
      \coordinate (ejex) at (0,0,\LONGEJEX);
      \coordinate (ejey) at (\LONGEJEY,0,0);
      \coordinate (ejez) at (0,\LONGEJEZ,0);
      % Ejes
      \draw[eje] (A1) -- (ejex) node[above left] {$x$};
      \draw[eje] (A1) -- (ejey) node[below=2pt] {$y$};
      \draw[eje] (A1) -- (ejez) node[left=2pt] {$z$};
      % Cubo
      % Cara trasera
      \filldraw[cara opaca] (A1) -- (A2) -- (A3) -- (A4) -- cycle;
      % Cara inferior
      \filldraw[cara opaca] (A1) -- (A4) -- (B4) -- (B1) -- cycle;
      % Cara izquierda
      \filldraw[cara opaca] (A1) -- (B1) -- (B2) -- (A2) -- cycle;
      % Cara derecha
      \filldraw[cara translucida] (A4) -- (B4) -- (B3) -- (A3) -- cycle;
      % Cara superior
      \filldraw[cara translucida] (A2) -- (A3) -- (B3) -- (B2) -- cycle;
      % Cara frontal
      \filldraw[cara translucida] (B1) -- (B2) -- (B3) -- (B4) -- cycle;
      % Vértices del cubo
      %\filldraw (A2) circle[radius=1pt] node[above left,black] {\scriptsize $A2$};
      %\filldraw (A3) circle[radius=1pt] node[above right,black] {\scriptsize $A3$};
      %\filldraw (A4) circle[radius=1pt] node[above right,black] {\scriptsize $A4$};
      %\filldraw (B1) circle[radius=1pt] node[left,black] {\scriptsize $B1$};
      %\filldraw (B2) circle[radius=1pt] node[left,black] {\scriptsize $B2$};
      %\filldraw (B3) circle[radius=1pt] node[right,black] {\scriptsize $B3$};
      %\filldraw (B4) circle[radius=1pt] node[right,black] {\scriptsize $B4$};
    \end{tikzpicture}
    \caption{}
    \label{fig:edcmat-cubo2}
  \end{figure}

  \vspace{-3ex}
  % ...........................................................................
  {\footnotesize \textcolor{gray}{[No resuelto]}}
  % \showSolved{res/matematicas}{matematicas-033.pdf}
  % ...........................................................................
  \medskip
  {\color{gray}
    \hrule
  }
  
  % --------------------------------------------------------------------------
  % Teorema de Stokes
  % --------------------------------------------------------------------------
\item Compruebe el teorema de Stokes para la función
  $\vvv{v} = (xy) \xhat{x} + (2yz) \xhat{y} + (3zx) \xhat{z}$.
  Utilizando el área triangular sombreada de la figura~\ref{fig:edcmat-triangulo1}.
  \vspace{-2ex}
  \begin{figure}[ht]
    \def\scl{.80}
    \def\lw{.8pt}
    \pgfmathsetmacro{\LONGEJEX}{2.6}
    \pgfmathsetmacro{\LONGEJEY}{2.2}
    \pgfmathsetmacro{\LONGEJEZ}{2.2}
    \pgfmathsetmacro{\LONGCATETOS}{1.5}
    \centering
    \begin{tikzpicture}[%
      scale=\scl,
      % tdplot_main_coords,
      eje/.style={->,line width=.6pt,black},
      trayecto/.style={-{Latex},line width=\lw,shorten >=1pt},
      cubo/.style={line width=.85pt,black},
      cara opaca/.style={line width=.85,fill=black!2,draw=black},
      cara translucida/.style={line width=.85,fill=black!2,draw=black,opacity=.75},
      ]
      % Coordenadas
      \coordinate (A1) at (0,0,0);
      \coordinate (A2) at (0,\LONGCATETOS,0);
      \coordinate (A3) at (\LONGCATETOS,\LONGCATETOS,0);
      \coordinate (A4) at (\LONGCATETOS,0,0);
      \coordinate (B1) at (0,0,\LONGCATETOS);
      \coordinate (B2) at (0,\LONGCATETOS,\LONGCATETOS);
      \coordinate (B3) at (\LONGCATETOS,\LONGCATETOS,\LONGCATETOS);
      \coordinate (B4) at (\LONGCATETOS,0,\LONGCATETOS);
      % Punto central del cubo
      \coordinate (O) at ($(A1)!.5!(B3)$);
      % Punto cara trasera
      \coordinate (tra) at ($(A2)!.5!(A4)$);
      % Punto cara izquierda
      \coordinate (izq) at ($(A2)!.5!(B1)$);
      % Punto cara superior
      \coordinate (sup) at ($(A2)!.5!(B3)$);
      % Punto cara derecha
      \coordinate (der) at ($(A3)!.5!(B4)$);
      % Punto cara frontal
      \coordinate (fro) at ($(B2)!.5!(B4)$);
      % Extremos de los ejes de coordenadas
      \coordinate (ejex) at (0,0,\LONGEJEX);
      \coordinate (ejey) at (\LONGEJEY,0,0);
      \coordinate (ejez) at (0,\LONGEJEZ,0);
      % Ejes
      \draw[eje] (A1) -- (ejex) node[above left] {$x$};
      \draw[eje] (A1) -- (ejey) node[below=2pt] {$y$};
      \draw[eje] (A1) -- (ejez) node[left=2pt] {$z$};
      % Triángulo
      \filldraw[fill=black!10,draw=black!10] (A2) -- (A1) -- (A4);
      \path[tips,-{Latex},line width=\lw] (A2) -- ($(A2)!.7!(A1)$);
      \draw(A2) -- (A1);
      \path[tips,-{Latex},line width=\lw] (A1) -- ($(A1)!.7!(A4)$);
      \draw (A1) -- (A4);
      \path[tips,-{Latex},line width=\lw] (A4) -- ($(A4)!.7!(A2)$);
      \draw (A4) -- (A2);
      %
      \node[below] at (A4) {$2$};
      \node[left] at (A2) {$2$};
      
      %\filldraw (A2) circle[radius=1pt] node[above left,black] {\scriptsize $A2$};
      %\filldraw (A3) circle[radius=1pt] node[above right,black] {\scriptsize $A3$};
      %\filldraw (A4) circle[radius=1pt] node[above right,black] {\scriptsize $A4$};
      %\filldraw (B1) circle[radius=1pt] node[left,black] {\scriptsize $B1$};
      %\filldraw (B2) circle[radius=1pt] node[left,black] {\scriptsize $B2$};
      %\filldraw (B3) circle[radius=1pt] node[right,black] {\scriptsize $B3$};
      %\filldraw (B4) circle[radius=1pt] node[right,black] {\scriptsize $B4$};
    \end{tikzpicture}
    \caption{}
    \label{fig:edcmat-triangulo1}
  \end{figure}

  \vspace{-3ex}
  % ...........................................................................
  {\footnotesize \textcolor{gray}{[No resuelto]}}
  % \showSolved{res/matematicas}{matematicas-034.pdf}
  % ...........................................................................
  \medskip
  {\color{gray}
    \hrule
  }
  
  % --------------------------------------------------------------------------
  % 
  % --------------------------------------------------------------------------
\item Compruebe que $\int (\vvv{\nabla}\times\vvv{v})\cdot d\vvv{a}$ depende sólo de
  la línea de contorno, y no de la superficie concreta que se use, usando la función
  $\vvv{v} = (2xz+3y^2) \xhat{y} + (4yz^2) \xhat{z}$ y la línea de contorno de la
  figura~\ref{fig:edcmat-linea-contorno1}.
  e integrando sobre las cinco
  caras del cubo de la figura~\ref{fig:edcmat-cubo3}. La cara de atrás en el plano $yz$
  está abierta.
  \vspace{-2ex}
  \begin{figure}[ht]
    \def\scl{.65}
    \pgfmathsetmacro{\LONGEJEX}{2.6}
    \pgfmathsetmacro{\LONGEJEY}{2.2}
    \pgfmathsetmacro{\LONGEJEZ}{2.2}
    \pgfmathsetmacro{\LONGCUBO}{1.5}
    \centering
    \begin{subfigure}[b]{.3\textwidth}
      \centering
      \begin{tikzpicture}[%
        scale=\scl,
        eje/.style={->,line width=.6pt,black},
        cubo/.style={line width=.85pt,black},
        cara opaca/.style={line width=.85,fill=black!2,draw=black},
        cara translucida/.style={line width=.85,fill=black!2,draw=black,opacity=.75},
        ]
        % Coordenadas
        \coordinate (A1) at (0,0,0);
        \coordinate (A2) at (0,\LONGCUBO,0);
        \coordinate (A3) at (\LONGCUBO,\LONGCUBO,0);
        \coordinate (A4) at (\LONGCUBO,0,0);
        \coordinate (B1) at (0,0,\LONGCUBO);
        \coordinate (B2) at (0,\LONGCUBO,\LONGCUBO);
        \coordinate (B3) at (\LONGCUBO,\LONGCUBO,\LONGCUBO);
        \coordinate (B4) at (\LONGCUBO,0,\LONGCUBO);
        % Punto central del cubo
        \coordinate (O) at ($(A1)!.5!(B3)$);
        % Punto cara trasera
        \coordinate (tra) at ($(A2)!.5!(A4)$);
        % Punto cara izquierda
        \coordinate (izq) at ($(A2)!.5!(B1)$);
        % Punto cara superior
        \coordinate (sup) at ($(A2)!.5!(B3)$);
        % Punto cara derecha
        \coordinate (der) at ($(A3)!.5!(B4)$);
        % Punto cara frontal
        \coordinate (fro) at ($(B2)!.5!(B4)$);
        % Extremos de los ejes de coordenadas
        \coordinate (ejex) at (0,0,\LONGEJEX);
        \coordinate (ejey) at (\LONGEJEY,0,0);
        \coordinate (ejez) at (0,\LONGEJEZ,0);
        % Ejes
        \draw[eje] (A1) -- (ejex) node[above left] {$x$};
        \draw[eje] (A1) -- (ejey) node[below=2pt] {$y$};
        \draw[eje] (A1) -- (ejez) node[left=2pt] {$z$};
        % Cubo
        % Cara trasera
        \filldraw[cara opaca] (A1) -- (A2) -- (A3) -- (A4) -- cycle;
        % Cara inferior
        % \filldraw[cara opaca] (A1) -- (A4) -- (B4) -- (B1) -- cycle;
        % Cara izquierda
        % \filldraw[cara opaca] (A1) -- (B1) -- (B2) -- (A2) -- cycle;
        % Cara derecha
        % \filldraw[cara translucida] (A4) -- (B4) -- (B3) -- (A3) -- cycle;
        % Cara superior
        % \filldraw[cara translucida] (A2) -- (A3) -- (B3) -- (B2) -- cycle;
        % Cara frontal
        % \filldraw[cara translucida] (B1) -- (B2) -- (B3) -- (B4) -- cycle;
        % Vértices del cubo
        % \filldraw (A2) circle[radius=1pt] node[above left,black] {\scriptsize $A2$};
        % \filldraw (A3) circle[radius=1pt] node[above right,black] {\scriptsize $A3$};
        % \filldraw (A4) circle[radius=1pt] node[above right,black] {\scriptsize $A4$};
        % \filldraw (B1) circle[radius=1pt] node[left,black] {\scriptsize $B1$};
        % \filldraw (B2) circle[radius=1pt] node[left,black] {\scriptsize $B2$};
        % \filldraw (B3) circle[radius=1pt] node[right,black] {\scriptsize $B3$};
        % \filldraw (B4) circle[radius=1pt] node[right,black] {\scriptsize $B4$};
      \end{tikzpicture}
      \caption{}
      \label{fig:edcmat-linea-contorno1}
    \end{subfigure}
    \begin{subfigure}[b]{.3\textwidth}
      \centering
    \begin{tikzpicture}[%
      scale=\scl,
      %tdplot_main_coords,
      eje/.style={->,line width=.6pt,black},
      %cubo/.style={white,double=black,double distance=.1mm,join=bevel,line width=1.2pt},
      cubo/.style={line width=.85pt,black},
      cara opaca/.style={line width=.85,fill=black!2,draw=black},
      cara translucida/.style={line width=.85,fill=black!2,draw=black,opacity=.75},
      ]
      % Coordenadas
      \coordinate (A1) at (0,0,0);
      \coordinate (A2) at (0,\LONGCUBO,0);
      \coordinate (A3) at (\LONGCUBO,\LONGCUBO,0);
      \coordinate (A4) at (\LONGCUBO,0,0);
      \coordinate (B1) at (0,0,\LONGCUBO);
      \coordinate (B2) at (0,\LONGCUBO,\LONGCUBO);
      \coordinate (B3) at (\LONGCUBO,\LONGCUBO,\LONGCUBO);
      \coordinate (B4) at (\LONGCUBO,0,\LONGCUBO);
      % Punto central del cubo
      \coordinate (O) at ($(A1)!.5!(B3)$);
      % Punto cara inferior
      \coordinate (inf) at ($(A1)!.5!(B4)$);
      % Punto cara izquierda
      \coordinate (izda) at ($(A2)!.5!(B1)$);
      % Punto cara superior
      \coordinate (sup) at ($(A2)!.5!(B3)$);
      % Punto cara derecha
      \coordinate (dcha) at ($(A3)!.5!(B4)$);
      % Punto cara frontal
      \coordinate (fro) at ($(B2)!.5!(B4)$);
      % Extremos de los ejes de coordenadas
      \coordinate (ejex) at (0,0,\LONGEJEX);
      \coordinate (ejey) at (\LONGEJEY,0,0);
      \coordinate (ejez) at (0,\LONGEJEZ,0);
      % Ejes
      \draw[eje] (A1) -- (ejex) node[above left] {$x$};
      \draw[eje] (A1) -- (ejey) node[below=2pt] {$y$};
      \draw[eje] (A1) -- (ejez) node[left=2pt] {$z$};
      % Vectores normales a las caras (menos a la de abajo)
      \def\radius{1.5pt}
      \def\linewidth{1.1pt}
      % Vector normal cara izquierda
      \draw[-{Latex},line width=\linewidth] (izda) -- +(left:3em)
      node[left] {\scriptsize\scshape (iv)};
      \filldraw[fill=black,draw=black] (izda) circle[radius=\radius];
      % Vector normal cara inferior
      %\draw[-{Latex},black,line width=\linewidth] ($(inf)+(0,.1,.1)$) -- ++($(inf)-1.6*(ejez)$)
      %node[right] {\scriptsize\scshape (ii)};
      \draw[-{Latex},black,line width=\linewidth] (inf) -- +(down:4em)
      node[right] {\scriptsize\scshape (ii)};
      \filldraw[fill=black,draw=black] ($(fro)+(0,.1,.1)$) circle[radius=\radius];
      %\draw[-{Latex},black,line width=\linewidth] (tra) -- ++($(fro)-.85*(ejex)$);
      %\filldraw[fill=black,draw=black] (tra) circle[radius=\radius];      
      % Cubo
      % Cara trasera
      \filldraw[cara opaca] (A1) -- (A2) -- (A3) -- (A4) -- cycle;
      % Cara inferior
      \filldraw[cara opaca] (A1) -- (A4) -- (B4) -- (B1) -- cycle;
      % Cara izquierda
      \filldraw[cara opaca] (A1) -- (B1) -- (B2) -- (A2) -- cycle;
      % Cara derecha
      \filldraw[cara translucida] (A4) -- (B4) -- (B3) -- (A3) -- cycle;
      % Cara superior
      \filldraw[cara translucida] (A2) -- (A3) -- (B3) -- (B2) -- cycle;
      % Cara frontal
      \filldraw[cara translucida] (B1) -- (B2) -- (B3) -- (B4) -- cycle;
      % Vector normal cara derecha
      \draw[-{Latex},line width=\linewidth] (dcha) -- +(right:3em)
      node[right] {\scriptsize\scshape (iii)};
      \filldraw[fill=black,draw=black] (dcha) circle[radius=\radius];
      % Vector normal cara frontal
      \draw[-{Latex},black,line width=\linewidth] ($(fro)+(0,.1,.1)$) -- ++($(fro)+.85*(ejex)$)
      node[above right=-4pt and 6pt] {\scriptsize\scshape (i)};
      \filldraw[fill=black,draw=black] ($(fro)+(0,.1,.1)$) circle[radius=\radius];
      % \draw[-{Latex},black,line width=\linewidth] (fro) -- ++($(fro)+.85*(ejex)$);
      % \filldraw[fill=black,draw=black] (fro) circle[radius=\radius];
      % Vector normal cara superior
      \draw[-{Latex},line width=\linewidth] (sup) -- +(up:3em)
      node[right] {\scriptsize\scshape (v)};
      \filldraw[fill=black,draw=black] (sup) circle[radius=\radius];
      
      % Vértices del cubo
      %\filldraw[fill=red,draw=red] (A2) circle[radius=1pt] node[above left,black] {\scriptsize $A2$};
      %\filldraw[fill=red,draw=red] (A3) circle[radius=1pt] node[above right,black] {\scriptsize $A3$};
      %\filldraw[fill=red,draw=red] (A4) circle[radius=1pt] node[above right,black] {\scriptsize $A4$};
      %\filldraw[fill=red,draw=red] (B1) circle[radius=1pt] node[left,black] {\scriptsize $B1$};
      %\filldraw[fill=red,draw=red] (B2) circle[radius=1pt] node[left,black] {\scriptsize $B2$};
      %\filldraw[fill=red,draw=red] (B3) circle[radius=1pt] node[right,black] {\scriptsize $B3$};
      %\filldraw[fill=red,draw=red] (B4) circle[radius=1pt] node[right,black] {\scriptsize $B4$};
    \end{tikzpicture}
    \caption{}
    \label{fig:edcmat-cubo3}
  \end{subfigure}
  \caption{}
\end{figure}

\vspace{-3ex}
% ...........................................................................
{\footnotesize \textcolor{gray}{[No resuelto]}}
% \showSolved{res/matematicas}{matematicas-035.pdf}
% ...........................................................................
\medskip
{\color{gray}
  \hrule
}

% --------------------------------------------------------------------------
% 
% --------------------------------------------------------------------------
\item Demuestre que
  \begin{subejercicio}
  \item
    \[
      \int_S f(\vvv{\nabla}\times\vvv{A})\cdot d\vvv{a}
      = \int_S [\vvv{A}\times (\vvv{\nabla}f)]\cdot d\vvv{a}
      + \oint_P f\vvv{A}\cdot d\vvv{l}
    \]
  \item
    \[
      \int_V \vvv{B}\cdot (\vvv{\nabla}\times\vvv{A})\, d\tau
      = \int_V [\vvv{A}\cdot (\vvv{\nabla}\times\vvv{B})]\, d\tau
      + \oint_S (\vvv{A}\times\vvv{B})\cdot d\vvv{a}
    \]    
  \end{subejercicio}

  % ...........................................................................
  {\footnotesize \textcolor{gray}{[No resuelto]}}
  % \showSolved{res/matematicas}{matematicas-036.pdf}
  % ...........................................................................
  \medskip
  {\color{gray}
    \hrule
  }
  
% --------------------------------------------------------------------------
% 
% --------------------------------------------------------------------------
\item Deduzca fórmulas para $r$, $\theta$, $\phi$ en términos de $x$, $y$, $z$,
  a partir de $x = r\sin\theta\cos\phi$, $y=r\sin\theta\sin\phi$, $z=r\cos\theta$.
  
  % ...........................................................................
  {\footnotesize \textcolor{gray}{[No resuelto]}}
  % \showSolved{res/matematicas}{matematicas-037.pdf}
  % ...........................................................................
  \medskip
  {\color{gray}
    \hrule
  }
  
% --------------------------------------------------------------------------
% 
% --------------------------------------------------------------------------
\item Exprese los vectores unitarios $\xhat{r}$, $\xhat{\theta}$, $\xhat{\phi}$ en términos de
  $\xhat{x}$, $\xhat{y}$, $\xhat{z}$. Obtenga, además, las expresiones inversas, esto es,
  los $\xhat{x}$, $\xhat{y}$, $\xhat{z}$, en función de $\xhat{r}$, $\xhat{\theta}$, $\xhat{\phi}$.
  
  % ...........................................................................
  {\footnotesize \textcolor{gray}{[No resuelto]}}
  % \showSolved{res/matematicas}{matematicas-038.pdf}
  % ...........................................................................
  \medskip
  {\color{gray}
    \hrule
  }

  % --------------------------------------------------------------------------
  % 
  % --------------------------------------------------------------------------
\item
  \begin{subejercicio}
  \item Compruebe el teorema de la divergencia para la función $\vvv{v}_1=r^2 \xhat{r}$,
    usando como volumen una esfera de radio $R$ centrada en el origen.
  \item Haga lo mismo para $\vvv{v}_2 = (1/r^2)\,\xhat{r}$.
  \end{subejercicio}
  
  % ...........................................................................
  {\footnotesize \textcolor{gray}{[No resuelto]}}
  % \showSolved{res/matematicas}{matematicas-039.pdf}
  % ...........................................................................
  %\medskip
  %{\color{gray}
  %  \hrule
  %}

  
  % #########################################################################################
  % #########################################################################################
  % #########################################################################################
  
  \clearpage
  \stepcounter{isubsheet} 
  
  % Añade 'Contenidos' al índice pdf.
  \pdfbookmark[2]{Hoja 6}{mathj06}
  
  
  % --------------------------------------------------------------------------
  % 
  % --------------------------------------------------------------------------
\item Calcule la divergencia de la función
  \[
    \vvv{v} = (r\cos\theta)\,\xhat{r} + (r\sin\theta)\,\xhat{\theta} + (r\sin\theta\cos\phi)\,\xhat{\phi}
  \]
  Compruebe el teorema de la divergencia para esta función, usando como volumen el cuenco semiesférico
  invertido de radio $R$, que descansa en el plano $xy$ y está centrado en el origen.
  Ver figura~\ref{fig:edcmat-semiesfera1}.
  \vspace{-2ex}
  \begin{figure}[ht]
    \def\scl{1}
    \def\lw{.8pt}
    \pgfmathsetmacro{\FACTOR}{.8}
    \pgfmathsetmacro{\LONGEJEX}{2.6}
    \pgfmathsetmacro{\LONGEJEY}{2.2}
    \pgfmathsetmacro{\LONGEJEZ}{2.2}
    \centering
   \begin{tikzpicture}[%
      scale=\scl,
      % tdplot_main_coords,
      eje/.style={->,line width=.6pt,black},
      trayecto/.style={-{Latex},line width=\lw,shorten >=1pt},
      cubo/.style={line width=.85pt,black},
      cara opaca/.style={line width=.85,fill=black!2,draw=black},
      cara translucida/.style={line width=.85,fill=black!2,draw=black,opacity=.75},
      ]
      % Coordenadas
      \coordinate (O) at (0,0,0);
      % Extremos de los ejes de coordenadas
      \coordinate (ejex) at (0,0,\FACTOR*\LONGEJEX);
      \coordinate (ejey) at (\FACTOR*\LONGEJEY,0,0);
      \coordinate (ejez) at (0,\FACTOR*\LONGEJEZ,0);
      % Ejes
      \draw[eje] (O) -- (ejex) node[above left] {$x$};
      \draw[eje] (O) -- (ejey) node[below=2pt] {$y$};
      \draw[eje] (O) -- (ejez) node[left=2pt] {$z$};
      % Elipse
      \draw[line width=.4pt, black!30] (O) ellipse (.8cm and .25cm);
      \coordinate (P) at ($(0,0) + (180:.8cm and .25cm)$);
      \draw[line width=.8pt] ($(0,0) + (0:.8cm and .25cm)$) (P) arc (180:360:.8cm and .25cm);
      % Semicircunferencia
      \draw[line width=.8pt] ($(0,0) + (.8cm,0)$) arc (0:180:.8cm);
    \end{tikzpicture}
    \caption{}
    \label{fig:edcmat-semiesfera1}
  \end{figure}

  \vspace{-3ex}
  % ...........................................................................
  {\footnotesize \textcolor{gray}{[No resuelto]}}
  % \showSolved{res/matematicas}{matematicas-040.pdf}
  % ...........................................................................
  \medskip
  {\color{gray}
    \hrule
  }
  
  % --------------------------------------------------------------------------
  % 
  % --------------------------------------------------------------------------
\item Calcule el gradiente y la laplaciana de la función $T = r(\cos\theta + \sin\theta \cos\phi)$.
  Compruebe la laplaciana pasando $T$ a coordenadas cartesianas y usando la ecuación
  \[
    \nabla^2
    = \frac{\partial^1 T}{\partial x^2}
    + \frac{\partial^2 T}{\partial y^2}
    + \frac{\partial^2 T}{\partial z^2}
  \]
  Compruebe el teorema del gradiente para esta función usando el camino mostrado
  en la figura~\ref{fig:edcmat-camino1}, que va desde $(0,0,0)$ hasta $(0,0,2)$.
  \vspace{-3ex}
  \begin{figure}[ht]
    \def\scl{.8}
    \def\lw{.8pt}
    \pgfmathsetmacro{\FACTOR}{.8}
    \pgfmathsetmacro{\LONGEJEX}{2.6}
    \pgfmathsetmacro{\LONGEJEY}{2.2}
    \pgfmathsetmacro{\LONGEJEZ}{2.2}
    \centering
   \begin{tikzpicture}[%
      scale=\scl,
      % tdplot_main_coords,
      eje/.style={->,line width=.6pt,black},
      trayecto/.style={-{Latex},line width=\lw,shorten >=1pt},
      cubo/.style={line width=.85pt,black},
      cara opaca/.style={line width=.85,fill=black!2,draw=black},
      cara translucida/.style={line width=.85,fill=black!2,draw=black,opacity=.75},
      ]
      % Coordenadas
      \coordinate (O) at (0,0,0);
      \coordinate (X) at (0,0,1);
      \coordinate (Y) at (1,0,0);
      \coordinate (Z) at (0,1,0);
      % Extremos de los ejes de coordenadas
      \coordinate (ejex) at (0,0,\FACTOR*\LONGEJEX);
      \coordinate (ejey) at (\FACTOR*\LONGEJEY,0,0);
      \coordinate (ejez) at (0,\FACTOR*\LONGEJEZ,0);
      % Ejes
      \draw[eje] (O) -- (ejex) node[left] {$x$};
      \draw[eje] (O) -- (ejey) node[below=2pt] {$y$};
      \draw[eje] (O) -- (ejez) node[left=2pt] {$z$};
      % 
      \draw[trayecto] (O) -- (X);
      \draw[trayecto] (X) arc (270:315:2cm and 1.3cm);
      \draw[trayecto] (1,0) arc (0:90:1);
      %
      \node[above left=-2pt and 0pt] at (X) {\footnotesize $2$};
      \node[below right=0pt and -2pt] at (Y) {\footnotesize $2$};
      \node[left] at (Z) {\footnotesize $2$};
    \end{tikzpicture}
    \caption{}
    \label{fig:edcmat-camino1}
  \end{figure}

  \vspace{-3ex}
  % ...........................................................................
  {\footnotesize \textcolor{gray}{[No resuelto]}}
  % \showSolved{res/matematicas}{matematicas-041.pdf}
  % ...........................................................................
  \medskip
  {\color{gray}
    \hrule
  }
  
  


% $\brcurs$ --> Vector separación.
% $\rcurs$ --> Módulo del vector separación.
% $\hrcurs$ --> Vector unitario separación.
%  $\hrcurs_{1}/\rcurs_{1}^{2}$

\end{ejercicio}


%%% Local Variables:
%%% coding: utf-8
%%% mode: latex
%%% TeX-engine: luatex
%%% TeX-master: "../edclasica-ej.tex"
%%% End:
