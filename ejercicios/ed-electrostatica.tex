% ed-electrostatica.tex
%
% Copyright (C) 2019-2025 José A. Navarro Ramón <janr.devel@gmail.com>
%
% ---------------------------------------------------------------------------
% ---------------------------------------------------------------------------
% HOJA
% ---------------------------------------------------------------------------
% ---------------------------------------------------------------------------
%\phantomsection
%\addcontentsline{toc}{subsection}{Hoja 1}
\setcounter{isubsheet}{1}

% Añade 'Contenidos' al índice pdf.
% \bookmark[level=2,dest=section]{Hoja 1}
% Nombre de enlace: 'cc1.2'
\pdfbookmark[2]{Hoja1}{elctrostatica1}

\begin{ejercicio}
% --------------------------------------------------------------------------
% 
% --------------------------------------------------------------------------
\item
\begin{subejercicio}
  \item  Se sitúan doce cargas iguales $q$ en los vértices de un polígono
    regular de doce lados (dodecágono).
    ¿Cuál es la fuerza neta sobre una carga de prueba $Q$ en el centro
    del polígono?
  \item Suponga que se elimina una de las doce cargas $q$.
    ¿Cuánto vale ahora la fuerza neta sobre $Q$?
  \item Ahora tenemos trece cargas iguales $q$ sobre los vértices de un
    polígono regular de trece aristas. ¿Cuál es la fuerza neta sobre una
    carga de prueba $Q$ en el centro del polígono?
  \item Si se retira una de las trece cargas $q$, ¿cuál sería ahora la
    fuerza neta sobre $Q$?
\end{subejercicio}

% ...........................................................................
  \showSolved{res/electrostatica}{electrostatica-001.pdf}
% ...........................................................................
\medskip
{\color{gray}
\hrule
}

% --------------------------------------------------------------------------
% 
% --------------------------------------------------------------------------
\item
  \begin{subejercicio}
    Calcule el campo eléctrio (módulo, dirección y sentido) a una distancia
    $z$ sobre el punto medio que hay entre dos cargas iguales y opuestas
    ($\pm q$) a una distancia $d$ entre ellas. La carga en la posición
    $x= +d/2$ es $-q$.
\end{subejercicio}

% ...........................................................................
  \showSolved{res/electrostatica}{electrostatica-002.pdf}
% ...........................................................................
\medskip
{\color{gray}
\hrule
}

% --------------------------------------------------------------------------
% 
% --------------------------------------------------------------------------
\item
\begin{subejercicio}
  Calcule el campo eléctrico a una distancia $z$ sobre el punto medio
  de un segmento rectilíneo de longitud $2L$ que tiene una densidad
  lineal de carga $\lambda$ uniforme.
\end{subejercicio}

% ...........................................................................
  \showSolved{res/electrostatica}{electrostatica-003.pdf}
% ...........................................................................
\medskip
{\color{gray}
\hrule
}


%  los vectores $\brcurs_{1}$, $\rcurs_{1}$, $\hrcurs_{1}$ y
%  $\hrcurs_{1}/\rcurs_{1}^{2}$



 
\end{ejercicio}


%%% Local Variables:
%%% coding: utf-8
%%% mode: latex
%%% TeX-engine: luatex
%%% TeX-master: "../edclasica-ej"
%%% End:
